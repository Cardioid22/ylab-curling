\documentclass[11pt,a4j]{jreport}

\usepackage{comment}
\usepackage{float}
\usepackage{color}
\usepackage{multicol}
\usepackage[dvipdfmx]{pict2e}
\usepackage{wrapfig}
\usepackage{graphicx}
\usepackage{bm}
\usepackage{url}
\usepackage{underscore}
\usepackage{colortbl}
\usepackage{tabularx}
\usepackage{fancyhdr}
\usepackage{ulem}
\usepackage{amsmath,amssymb,amsfonts}
\usepackage{algorithmic}
\usepackage{textcomp}
\usepackage{xcolor}
\usepackage[ipaex]{pxchfon}
\usepackage{algorithmic}
\usepackage{algorithm}
\usepackage{cite}

\usepackage[top=30truemm,bottom=30truemm,left=25truemm,right=25truemm]{geometry}
\setlength{\headheight}{15.5pt} % エラーで指定されている値よりも大きな値を設定する
\addtolength{\topmargin}{-3.5pt} % 必要に応じて topmargin も調整する


\begin{document}

\thispagestyle{empty}
\begin{center}

\vspace{20mm}
{\Large\noindent 2025年度 卒業論文}\\
\vspace{40mm}
{\huge\noindent\textbf{連続行動空間におけるMCTSでのクラスタリング導入による探索効率化手法の提案
}}\\
\medskip
{\huge\noindent\textbf{論文タイトル(2行目)}}\\
\vspace{60mm}

%ここの空白文字は適当に
{\Large\noindent
2026年1月31日\\
\vspace{\baselineskip}
  所属 明治大学 \\
\vspace{\baselineskip}
指導教員 横山大作    \\
\vspace{\baselineskip}
 学籍番号 157R227127\\
\vspace{\baselineskip}
  名前 仲亜斗夢\\
}
\vspace{40mm}

\end{center}

\thispagestyle{empty}
\clearpage

%=====================================================================================
\renewcommand{\abstractname}{要旨}

\begin{abstract}
研究の要旨。なんやかんやなんやかんやなんやかんやなんやかんやなんやかんやなんやかんやなんやかんやなんやかんやなんやかんやなんやかんやなんやかんやなんやかんやなんやかんやなんやかんやなんやかんやなんやかんやなんやかんやなんやかんやなんやかんやなんやかんやなんなんや
\end{abstract}

%=====================================================================================

% 目次の表示
\tableofcontents

%=====================================================================================
\pagestyle{fancy}
\lhead{\rightmark}
\renewcommand{\chaptermark}[1]{\markboth{第\ \normalfont\thechapter\ 章~~#1}{}}
%=====================================================================================

\chapter{はじめに} %章


\section{研究背景} %1.1
\subsection{デジタルカーリングとは} %1.1.1
デジタルカーリングとは、コンピュータ上の物理シミュレータを用いた仮想的なカーリングスペースであり、カーリングの戦略を議論・検証するためのプラットフォームとして提案されたものである\cite{uec_cup_report}。伊藤らによる第1回UEC杯デジタルカーリング大会報告によると、本システムは二人零和有限確定完全情報ゲームとしての性質を持ちつつも、投球結果に確率的な誤差が含まれる「不確定ゲーム」として定義される。

\subsubsection*{システムの概要と物理演算}
デジタルカーリングでは、2次元物理演算エンジン「Box2D」を用いてストーンの挙動をシミュレートしてい。ストーンの軌道計算においては、プレイヤーが指定した初速度ベクトルと回転方向に対し,正規分布に従うランダムなノイズ(外乱)が加算される。これにより、現実のカーリングにおける「氷の状態変化」や「投球のブレ」に相当する不確定性が再現されている。

\subsubsection*{基本ルールと試合の進行}
試合の進行およびルールは、実際のカーリング競技に準拠している.
\begin{itemize}
    \item \textbf{試合構成}: 通常10エンド(または8エンド)で行われ、各エンドにおいて先攻・後攻の2チームが交互に8投ずつ、計16投のストーンを投げる。
    \item \textbf{得点計算}: 全投球終了後、ハウス(円)の中心(ティー)に最も近いストーンを持つチームが相手チームのNo.1ストーンよりも内側にある自チームのストーンの数だけ得点を得る。敗北したチームの得点は0点となる。
    \item \textbf{手番の決定}: 第1エンドはコイントス等で決定し、第2エンド以降は前のエンドで得点したチームが先攻となる。両チーム無得点(ブランクエンド)の場合は、手番は交代しない.
    \item \textbf{無効となるストーン}: ホッグラインを超えなかったストーン、バックラインを完全に超えたストーン、サイドラインに接触したストーンはプレイエリアから除外される。
\end{itemize}

これらの仕様により、デジタルカーリングは単なる物理シミュレーションにとどまらず、不確定性リスクを管理しながら最適な着手を選択する高度な戦略性が求められるゲーム環境となっている。
% 画像ほしいかも。


\section{研究目的}
近年、シミュレーション環境を用いた競技型AIの研究が盛んに行われており、特にデジタルカーリングのように行動空間が連続で構成されるゲームにおいては、高度な意思決定アルゴリズムが求められている。モンテカルロ木探索(Monte Carlo Tree Search, MCTS)は多くの分野で成功を収めているが、連続行動空間では探索の効率が著しく低下するという課題がある。特に大渡ら[1]の研究や加藤ら[2]の研究では、デジタルカーリングシステムにおけるゲーム木探索の有効性が示される一方で、候補手の絞り込みや、連続行動空間における最適な手の選択が依然として課題であることが明記されている。
一方、連続空間への対応としては深層強化学習も注目されているが、これらは学習コストが高く、設計も複雑であり、特定の環境に依存しやすいという欠点を持つ。従来のMCTSにおいても、グリッドによる固定的な離散化が試みられているが、局面ごとの柔軟性に欠け、探索精度を犠牲にしてしまう恐れがある。本研究はこれらの先行研究の課題を考慮したうえで、行動空間の離散化に加えてクラスタリングを導入し、局面に応じて動的に有効な候補手を抽出・選別することで、探索の深さと精度を両立する手法を提案する。これにより、限られた計算資源の中でも有効な手の選択が可能となり、デジタルカーリングAIの発展に寄与することを目的とする。

\section{本論文の構想}
近年、デジタルカーリングの分野ではMCTSを基盤としたAIプログラムが公式大会において高い競技成績を残しており、その有効性が実証されている。例えば、第10回UEC杯では大渡が開発したMCTSを用いた「歩」が優勝し[3]、さらに加藤らによって開発されたExpectimax-searchに基づく「じりつくん」シリーズは複数の大会で上位に入賞しており[4]、第11回大会では改良版である「Jiritsukun-Jr」が優勝を果たしている[5]。これらの成果は、MCTSがデジタルカーリングにおいて実践的かつ強力な手法であることを示している。
しかし、MCTSを連続行動空間であるデジタルカーリング環境[6]にそのまま適用するには課題が残されている。具体的には、無限に存在する候補手の中から探索対象を選定する必要があるため、計算資源が広く分散し、探索の効率が著しく低下してしまうという問題である。実際大渡らや加藤らの研究においても、候補手の絞り込みや行動の最適化が今後の課題として指摘されている。本研究では、連続行動空間を離散化し、さらにクラスタリングによって類似した行動を統合・代表化することで、効率よく有効な候補手を選定する手法を提案する。その有効性を検証するために、クラスタリングを導入したMCTS(以後、提案手法)と、クラスタリング無しの従来のMCTSとの比較実験を行う。評価は、①探索木の成長傾向、②候補手の選択分布、③対戦成績の3点から行う。
まず、各手法が生成する探索木を比較し、クラスタリングにより選ばれた代表候補手(集合K)のうち、実際に訪問回数の多かった手の割合を測定する。これにより、クラスタリングが有効な手を適切に抽出できたかを評価する。
次に、両手法を無制限時間下で対戦させ、その勝敗結果を性能指標とする。提案手法では、候補手集合Cの一部要素を代表手集合Kの要素に置き換えて、探索深さ3でプレイアウトを行い、選択された手の分布も記録する。
最後に、実際に両手法を対戦させ、提案手法が従来手法に比べて探索効率や精度の点で優れているかを評価し、その有効性を明らかにする。

%これまでに調べた大事な技法は章単位で書いて良い
%ex. モンテカルロ, NN, 色彩について

\chapter{関連研究}
%調べたことは第1章よりもこっちの方が良い。
%研究目的に必要な情報以外はこっちに書くイメージ
[1]	大渡勝己, 田中哲朗:カーリングAIに対するモンテカルロ木探索の適用, ゲームプログラミングワークショップ2016論文集, pp. 180-187(2016).
[2]	加藤修, 飯塚博幸, 山本雅人:不確定性を含むデジタルカーリングにおけるゲーム木探索, 情報処理学会論文誌ジャーナル, Vol.57, No.11, pp.2354-2364(2016).
[4]	伊藤毅志, 森健太郎:デジタルカーリング大会報告2015年度, 情報処理学会研究報告, 2016-GI-36, No.2, pp. 1 - 5(2016).
[8]	佐藤佳州, 高橋大介:モンテカルロ木探索によるコンピュータ将棋, 情報処理学会論文誌, Vol.50, No.11, pp. 2740-2751(2019).
%http://minerva.cs.uec.ac.jp/cgi-bin/curling/wiki.cgiデジタルカーリング大会報告2015年度, 情報処理学会研究報告
デジタルカーリングに関する研究は、ゲームの不確定性や連続空間という特性を扱うためにいくつかのアプローチが提案されている。

大渡ら[1]は、連続状態空間を持つデジタルカーリングに対して、モンテカルロ木探索(MCTS)を適用する手法を提案している。彼らの手法では、連続的な状態空間を階層的に分割する「状態木(State Tree)」という構造を導入することで、無限に存在する状態を有限のノード集合として扱い、MCTSによる探索を可能にした。実験の結果,単純なシミュレーション方策を用いた場合と比較して、提案手法が有効であることを示しており、連続空間における探索において適切な離散化や状態の抽象化が重要であることを示唆している。

一方、加藤ら[2]は不確定性を含むデジタルカーリングに対してExpectimax法によるゲーム木探索を提案している。彼らは投球目標座標と回転方向の組み合わせを候補手とし、実行時に加わる外乱(不確定性)を考慮して、確率的に遷移するチャンスノードをゲーム木に導入した。不確定性を考慮しない場合と考慮する場合の比較実験を行い、不確定性を考慮して先読みを行うことで、既存の強力なAIプログラムである「GCCS」に対しても高い勝率を達成できることを示した。

これらの先行研究は、デジタルカーリングにおける「連続空間の離散化」や「不確定性の考慮」が極めて重要であることを示している。本研究ではこれらの知見を踏まえつつ、候補手の生成においてクラスタリングを用いることで、より効率的に有効な手を探索する手法について議論する。

\chapter{作成した分析用ツール}
研究の過程で、シミュレーションの局面状態を視覚的に確認するためのツールを作成した。デジタルカーリングの研究では、数値データ(ストーンの座標 $(x, y)$)だけでは戦況を直感的に把握することが困難である。そこで、PythonおよびMatplotlibを用いて、ストーン情報を入力するだけでカーリングシート上の配置を可視化するスクリプトを実装した。

本ツールは以下の機能を持つ。
\begin{itemize}
    \item \textbf{シートの描画}: 実際の競技ルールに基づいた正確な寸法で、ハウス(12ft, 8ft, 4ft, Button)、ホッグライン、ティーラインなどを描画する。
    \item \textbf{ストーンの配置}: 任意の個数のストーンについて、座標と所属チーム(先攻・後攻)を指定することで、シート上にプロットする。
    \item \textbf{チームの識別}: チーム0(先攻)を黄色、チーム1(後攻)を赤色で色分けし、視認性を高めている。
\end{itemize}

このツールを用いることで、MCTSが生成した「理想的なショット」や、実験で用いる「特定のテストケース(局面)」がどのような状況であるかを即座に可視化・画像化することが可能となり、実験結果の分析やデバッグ効率が大幅に向上した。

%ここにシートに変換される様子が分かる画像を貼る
\begin{figure}[H]
 \begin{minipage}{0.5\textwidth}
  \centering
  \makeatletter\def\@captype{table}\makeatother
  \caption{ストーンの配置サンプルデータ}
  \begin{tabular}{|l|r|r|r|} \hline
   チーム & ストーンID & $x$座標 & $y$座標 \\ \hline
   0(先行) & 0 & 0.0 & 38.405 \\
   1(後攻) & 1 & -1.0 & 39.0 \\ 
   0(先行) & 2 & -1.0 & 37.810 \\ 
   1(後攻) & 3 & 1.0 & 38.405 \\ \hline
  \end{tabular}
 \end{minipage}
 \hfill
 \begin{minipage}{0.46\textwidth}
  \centering
  \includegraphics[width=\textwidth]{figure/data_to_stone_positions.png}
  \caption{表3.1の可視化結果}
 \end{minipage}
\end{figure}

\chapter{ルールベース+MCTSプレイヤーの作成}
本章では、提案手法の比較対象(ベースライン)として作成したルールベース戦略をモンテカルロ木探索(MCTS)に組み込んだプレイヤーについて述べる。

\section{ルールベースの規則}
本プレイヤーは、カーリングの定石に基づいた以下の5つの基本戦略(ルール)を実装しており、局面に応じてこれらを選択肢として生成する。

\begin{enumerate}
    \item \textbf{ガード戦略}: 
    ハウス手前にガードストーンを配置し、自チームのストーンを守るときや相手の攻撃ストーンの進路を遮る戦略。序盤や、相手が有利な状況で多用される。先行の場合はセンターガード、後攻の場合はコーナーガードを選択する。
    
    \item \textbf{重心戦略}: 
    ハウス内にある相手ストーン群の重心を計算し、その位置を狙う戦略。相手ストーンが散らばっている場合に、まとめて一掃したり、その裏に隠れたりするために用いる。局面を意図的に崩した方が自チームが優位になる場合に使用される。
    
    \item \textbf{ドロー戦略}: 
    ハウスの中心(ボタン)を狙ってストーンを投げる、最も基本的な戦略。得点を確保するために重要であり、かつ、強い一手である。
    
    \item \textbf{テイクアウト戦略}: 
    ハウス内に存在する相手のNo.1ストーン(最も中心に近い石)を狙い、弾き出す戦略。相手の得点を防ぐために有効である。
    
    \item \textbf{高速球戦略}: 
    非常に速い速度($v_y \approx 3.0$ m/s)でストーンを投げ、ハウス内の石を乱したり、ガードを強引に弾き飛ばしたりする攻撃的な戦略。終盤でリスクを取って局面を打開する際に選択される。
\end{enumerate}

\section{MCTSのノードの選択方法}
本手法におけるノード展開では、前述の5つの戦略から候補手を選定する。すべての戦略を均等に探索するのではなく、エンドの進行状況(投球数)に応じて候補を絞り込むヒューリスティックを導入している。

\begin{itemize}
    \item \textbf{序盤 (ショット数 $< 5$)}: 
    先行の場合なら最初の3投、後攻の場合なら最初の2投の投球では、場を整えることが重要であるため、強制的に\textbf{ガード戦略}を選択する。このフェーズでは探索による勝利確率ではなく、ガードの位置評価関数(センターラインへの近さやカバー率)に基づくスコア(\texttt{EvaluateGuardStone})を用いてUCT値を計算する。
    
    \item \textbf{中盤以降}: 
    続いて6投目までは、攻撃の選択肢を増やす。
    \begin{itemize}
        \item ショット数 $ < 11$: \textbf{重心、ドロー、テイクアウト}からランダムに選択。
        \item ショット数 $\ge 11$: \textbf{高速球}を含めた全攻撃戦略からランダムに選択。
    \end{itemize}
\end{itemize}

また、探索の多様性を確保するため、親ノードで選択された戦略と同じ戦略を連続して選ばないように制約を加えている(\texttt{previous\_choice}による排他制御)。


\section{局面評価関数の実装}
MCTSのシミュレーションにおいて、各局面の良し悪しを判定するために、ヒューリスティックな評価関数(\texttt{EvaluateBoard})を実装した。この関数は、単にハウス内ストーンの得点計算を行うだけでなく、有利な状況を数値化して評価値 $S$ を算出する。具体的な評価項目は以下の通りである。

\begin{itemize}
    \item \textbf{距離スコア}: ハウス内にある自チームの各ストーンについて、中心からの距離に応じたスコアを加算する。
    \[ S_{dist} = \sum_{s \in \text{MyStones}} \max(0, 4.0 - \text{dist}(s, \text{Center})) \]
    \item \textbf{ストーン数優位}: ハウス内の有効ストーン数が相手より多い場合、加点($+3.0$)する。
    \item \textbf{ガードストーン評価}: ハウス手前に配置されたガードストーンについて、自チームのものは加点($+2.0$)、相手チームのものは減点($-2.0$)とする。
    \item \textbf{No.1ストーンボーナス}: 自チームのストーンがハウス中心に最も近い(No.1ストーンである)場合、大きなボーナス点($+20.0$)を加算する。
    \item \textbf{弾き出しボーナス}: 直前の状態と比較して、相手ストーンを弾き出した(ハウス外に出した)場合や、自チームのストーンが増えた場合にボーナスを加算する。
\end{itemize}

最終的な評価値はこれらの合計として算出され、MCTSはこの値に基づいて有利な分岐を探索する。

\section{対戦結果}
本手法を用いて、電気通信大学が主催する第10回GAT (Game AI Tournaments) デジタルカーリング部門に参加した。その結果、参加10チーム中9位という成績に終わった。
以下に、この結果に至った敗因と、大会を通じて得られた課題について述べる。

\subsection{探索不足による精度の低下}
最大の課題は、思考時間の制約によりMCTSの十分な探索数を確保できなかった点である。デジタルカーリングは連続空間におけるゲームであり、正確なショットを選択するためには膨大な数のシミュレーションが不可欠である。しかし、実時間制約(1手あたりの制限時間)を守るために探索回数(Iteration)を制限せざるを得ず、その結果、読みの深さが不足し、相手の戦略に対する有効な対抗手を発見できないケースが多発した。

\subsection{ルールベースによる攻撃の硬直化}
また、候補手生成をルールベースに依存していたことも敗因の一つである。本手法では、あらかじめ定義された5つの戦略(ガード、ドロー、テイクアウト等)の中からしか手を選べないため、局面の微妙な変化に対応した柔軟なショット(例えば、少しずらして複数の石に当てるショットや、複雑なコンビネーションショット等)を生成することができなかった。
これにより、攻撃のバリエーションが限定され、相手に戦略を読まれやすい単調なプレイ展開となってしまった。これらの経験から、ルールベースの候補手生成だけでは限界があり、より多様な候補手を探索空間から自律的に抽出する仕組みが必要であることを認識した。


\chapter{クラスタリング+MCTSプレイヤーの提案}
\section{クラスタリングの目的}
連続解空間において候補手を選定することは依然として課題となっている。それらの候補手を有限の範囲に絞って全探索することは大量の計算リソースを必要とする。それを改善する方法として本研究ではクラスタリングを用いる。クラスタリングによって大量にある候補手から類似手をまとめることで、探索空間を効率的に削減することを目的とする。
\section{実装方法}
本研究で提案するクラスタリングを用いた候補手生成およびMCTSへの適用手法について、以下の5つの手順で説明する。

\begin{enumerate}
    \item \textbf{ハウス周辺の座標を離散化する(グリッド化)} \\
    まず、ハウスの中心周辺の領域を$M \times N$のグリッドに分割し、離散的な着手点を作成する。これにより、本来は連続的で無限に存在する着手位置を、有限個の代表的なターゲット位置の集合として扱うことが可能となる。

    \item \textbf{グリッドの座標に対して、試し投げ後の局面を記録する} \\
    次に、グリッドの各交点(ターゲット位置)に向けて、事前に算出した適切な初速度ベクトルを用いてシミュレーション(試し投げ)を行う。このシミュレーションはデジタルカーリングが提供する環境に基づいて行う。この結果得られた遷移後の局面状態(全ストーンの位置情報)を、各ターゲット位置に対応する「投球後の予測状態」としてすべて記録する。

    \item \textbf{投げ終わった後の局面の状態でグループ化を行う} \\
    記録された多数の予測状態に対して、局面の類似性に基づくクラスタリングを行う。本手法では、局面の特徴を適切に捉えるために、以下の特徴抽出と2段階のクラスタリングプロセスを採用している。

    まず、各局面から「特徴ベクトル」を抽出する。特徴量としては、局面上の総ストーン数、ハウス周辺を6つの領域(左上・中上・右上・左下・中下・右下)に分割した際の各領域における自チーム・相手チームのストーン数、ハウス内のストーン数、およびNo.1ストーン(最も中心に近い石)の所属チーム情報を用いる。

    クラスタリングは以下の手順で実行される。
    \begin{enumerate}
      \item \textbf{粗分類}: 局面上の「総ストーン数」が一致するもの同士でグループ分けを行う。これにより、シミュレーション結果としてストーン数が変化したもの(テイクアウト成功時など)とそうでないものを効率的に分離する。
      \item \textbf{詳細分類}: 各グループ内で、k-means法を用いてさらに細分化する。ここでの類似度(距離)計算には、単純なユークリッド距離ではなく、領域ごとのストーン分布の差、ハウス内のストーン数の差、No.1ストーンの不一致度などを重み付けして加算した指標を用いる。これにより、戦況の構造(有利不利の状況や石の配置バランス)が似ている局面同士を同一のクラスタにまとめ上げる。
    \end{enumerate}

    \item \textbf{各クラスタの代表点を選出し、MCTSの子ノードとする} \\
    生成された各クラスタの中から、そのクラスタを代表する一つの状態を選出する。クラスタ内に実在する状態の中から選出する。
    選出基準としては、静的な局面評価関数を用いて各状態のスコアを算出し、自チームにとって最も評価値(有利度)が高い状態を代表点として採用する。この代表状態に対応するショット(初速度ベクトル)が次の探索ステップにおける有力な候補手となる。

    \item \textbf{MCTSの探索結果より最善手を選出する} \\
    選出された各クラスタの代表手を、MCTSのルートノード(自チームにおける最新の状態)直下の子ノードとして展開する。これにより、グリッド化によって生成された大量(数千通り)の候補手を、クラスタ数と同程度(数個〜数十個)の「有望な手のパターン」にまで絞り込むことができる。
    MCTSはこの絞り込まれた代表ノードに対して重点的にシミュレーションを行い、UCBなどの基準に従って探索木を成長させる。最終的に、最も訪問回数が多く、勝率が高いと判断された代表手を選択し、実際の手として実行する。
\end{enumerate}

    \begin{figure}[H]
        \centering
        \includegraphics[width=0.4\linewidth]{figure/shot_grid_visualization.png}
        \caption{着手点のグリッド化 ($M=4, N=4$)}
        \label{fig:shot_grid}
    \end{figure}

    \begin{figure}[H]
        \centering
        \includegraphics[width=0.4\linewidth]{figure/clustering_regions.png}
        \caption{クラスタリングにおける6つの領域分割}
        \label{fig:clustering_regions}
    \end{figure}

\chapter{クラスタリング+MCTSプレイヤーの評価}
本研究では、提案するクラスタリング手法の有効性を検証するために、「AllGrid MCTS」と「Clustered MCTS」の比較実験を行った。
AllGrid MCTSは、グリッド上の全ての着手(候補手)を子ノードとして展開する探索木を構築する手法である。これは、探索空間を間引くことなく網羅的に探索するため、十分な計算時間を与えれば、グリッド化された条件下における理論上の最善手に収束すると考えられる。本実験では、このAllGrid MCTSによって導き出された解を「正解」とみなす。

一方、Clustered MCTSは、前章で述べた通り、全グリッド状態をクラスタリングし、各クラスタの代表手のみを子ノードとして展開する手法である。この手法は、探索の幅を劇的に削減できるため、限られた計算資源での探索効率向上が期待される。

実験では、同一の局面に対して両手法を実行し、それぞれが選択した最終的な候補手(ベストショット)がどの程度一致するかを比較することで、クラスタリングによる代表手選出の妥当性を評価する。

\section{評価指標の定義}
一致率の評価として、本研究では以下の2つの指標を定義する。

\begin{itemize}

    \item \textbf{完全一致率(Exact Agreement Rate)}: \\
    AllGrid MCTSが選択した候補手と、Clustered MCTSが選択した候補手が全く同一のグリッドID(ターゲット座標)である割合。これが高いほど、クラスタリングを行っても情報の損失がなく、本来の最善手をピンポイントで特定できていることを意味する。
    
    \item \textbf{クラスタ一致率(Cluster Agreement Rate)}: \\
    AllGrid MCTSが選択した候補手が、Clustered MCTSが選択した候補手と同じ「クラスタ」に属している割合。
    カーリングにおいては、数cm単位の厳密な座標の一致よりも、「ガードの裏に隠す」「ハウス内の特定のエリアに置く」といった戦術的な意図が合致しているかが重要である。したがって、完全一致率が低くても、クラスタ一致率が高ければ、戦略的に妥当な判断ができていると評価できる。
    
\end{itemize}

\section{実験条件}
評価実験には、実際の試合で頻出する様々な局面や戦略的判断が分かれる局面を想定したテストケースを用いた。
具体的には、以下の10種類のジャンルを用意し、各ジャンルにつき配置を微調整した10通りのバリエーションを作成した。合計100ケース(10ジャンル $\times$ 10バリエーション)の局面に対して実験を行った。

使用した局面パターンのジャンルは以下の通りである。
\begin{enumerate}
    \item \textbf{CenterGuard}: センターライン上にガードストーンが配置された状況。ハウス中心への直通コースを塞ぐことで、相手のドローショットを牽制したり、自チームが後に中心を攻めるための拠点とする防御的な局面である。
    \item \textbf{CornerGuards}:  ハウスの端(コーナー)を守る位置にガードストーンが配置された状況。ハウス中心を空けておくことで複数得点を狙う攻撃的な展開(ブランクエンド狙いなど)への布石となる局面である。
    \item \textbf{SingleDraw}: ハウス内にストーンが1つだけ存在するシンプルな配置。そのストーンに対するテイクアウトや、フリーズ、あるいは無視して別の場所に置くなど、基礎的な着手選択の精度が問われる局面である。
    \item \textbf{DoubleDraw}: ハウス内に2つのストーンが配置された状況。ダブルテイクアウト(2つ同時に弾き出す)を狙うか、片方だけを処理するかなど、複数の石に対する相互作用を考慮する必要がある局面である
    \item \textbf{HouseCorners}: ウス内の四隅にストーンが散らばっている状況。中心から離れた位置にあるストーンに対して、曲がり幅の大きいショットや正確なウェイト配分でアプローチできるかが試される局面である。
    \item \textbf{GuardAndDraw}: ハウス前のガードストーンとハウス内のドローストーンが組み合わさった配置。ガードを回避して中の石を狙うカムアラウンドショットや、ガードを利用したランバックショットなど、高度なライン取りが要求される局面である。
    \item \textbf{Random}: シート全域にランダムにストーンを配置した状況。定石や典型的なパターンには当てはまらない予期せぬ局面において、汎用的な対応能力や探索のロバスト性を検証するための局面である。
    \item \textbf{FreezeAttempt}: 相手のストーンに対してフリーズ(接触させて止める)ショットが有効となる配置。相手の石を壁として利用し、自らの石をテイクアウトから守るような、高精度な配置戦略が求められる局面である。
    \item \textbf{Corner}:  ハウスの左右の端付近で攻防が行われている配置。センターストーンとは異なり、石が外に滑り出やすい位置であるため、非常に繊細なウェイトコントロールとカール制御が重要となる局面である。
    \item \textbf{Crowded}: ハウス内外に多数のストーンが密集している混戦状態。一度のショットで複数の石が動く可能性が高く、物理挙動の連鎖(玉突き)を正確に予測する深い読みが必要となる局面である。
\end{enumerate}

\begin{figure}[H]
    \centering
    \begin{minipage}{0.48\textwidth}
        \centering
        \includegraphics[width=0.9\linewidth]{figure/CenterGuard_v0.png}
        \caption{CenterGuard}
    \end{minipage}
    \hfill
    \begin{minipage}{0.48\textwidth}
        \centering
        \includegraphics[width=0.9\linewidth]{figure/CornerGuards_v0.png}
        \caption{CornerGuards}
    \end{minipage}
\end{figure}

\begin{figure}[H]
    \centering
    \begin{minipage}{0.48\textwidth}
        \centering
        \includegraphics[width=0.9\linewidth]{figure/SingleDraw_v0.png}
        \caption{SingleDraw}
    \end{minipage}
    \hfill
    \begin{minipage}{0.48\textwidth}
        \centering
        \includegraphics[width=0.9\linewidth]{figure/DoubleDraw_v0.png}
        \caption{DoubleDraw}
    \end{minipage}
\end{figure}

\begin{figure}[H]
    \centering
    \begin{minipage}{0.48\textwidth}
        \centering
        \includegraphics[width=0.9\linewidth]{figure/HouseCorners_v0.png}
        \caption{HouseCorners}
    \end{minipage}
    \hfill
    \begin{minipage}{0.48\textwidth}
        \centering
        \includegraphics[width=0.9\linewidth]{figure/GuardAndDraw_v6.png}
        \caption{GuardAndDraw}
    \end{minipage}
\end{figure}

\begin{figure}[H]
    \centering
    \begin{minipage}{0.48\textwidth}
        \centering
        \includegraphics[width=0.9\linewidth]{figure/Random_v6.png}
        \caption{Random}
    \end{minipage}
    \hfill
    \begin{minipage}{0.48\textwidth}
        \centering
        \includegraphics[width=0.9\linewidth]{figure/FreezeAttempt_v6.png}
        \caption{FreezeAttempt}
    \end{minipage}
\end{figure}

\begin{figure}[H]
    \centering
    \begin{minipage}{0.48\textwidth}
        \centering
        \includegraphics[width=0.9\linewidth]{figure/Corner_v0.png}
        \caption{Corner}
    \end{minipage}
    \hfill
    \begin{minipage}{0.48\textwidth}
        \centering
        \includegraphics[width=0.9\linewidth]{figure/Crowded_v0.png}
        \caption{Crowded}
    \end{minipage}
\end{figure}

\section{AllGrid v.s. Clusteredの対比実験}
各テストケースにおいて、AllGrid MCTSには十分な探索回数を与えて実行し、その結果を基準解とした。これに対し、Clustered MCTSは探索回数(イテレーション数)を段階的に変化させて実行し、少ない探索回数でどの程度基準解に近づけるか(一致率の推移)を検証した。
実験は全て同一の物理シミュレータ設定の下で行い、偶然性を排除するために十分な試行回数を確保した上で平均的な挙動を分析した。

\section{実験結果}


\chapter{まとめ}
研究のまとめ。なんやかんやなんやかんやなんやかんやなんやかんやなんやかんやなんやかんやなんやかんやなんやかんやなんやかんやなんやかんやなんやかんやなんやかんやなんやかんやなんやかんやなんやかんやなんやかんやなんやかんやなんやかんやなんやかんやなんやかんや

%=====================================================================================
% \chapter*{謝辞} %章を付けずにタイトル表示
% \addcontentsline{toc}{chapter}{謝辞} %章立てせずに目次に追加するおまじない
% 本論文を作成するにあたり、---- みなさまに感謝の意を表します.


%=====================================================================================

\addcontentsline{toc}{chapter}{参考文献} %章立てせずに目次に追加するおまじない
\renewcommand{\bibname}{参考文献} %これがないと,タイトルが「関連図書」になってしまう
\bibliographystyle{junsrt} %本文に\cite{}を入れることで,参考文献表示
\bibliography{refer} %bibtexファイルの読み込み


\end{document}

