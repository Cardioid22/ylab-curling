\documentclass[11pt,a4j]{jreport}

\usepackage{comment}
\usepackage{float}
\usepackage{color}
\usepackage{multicol}
\usepackage[dvipdfmx]{pict2e}
\usepackage{wrapfig}
\usepackage{graphicx}
\usepackage{bm}
\usepackage{url}
\usepackage{underscore}
\usepackage{colortbl}
\usepackage{tabularx}
\usepackage{fancyhdr}
\usepackage{ulem}
\usepackage{amsmath,amssymb,amsfonts}
\usepackage{algorithmic}
\usepackage{textcomp}
\usepackage{xcolor}
\usepackage[ipaex]{pxchfon}
\usepackage{algorithmic}
\usepackage{algorithm}
\usepackage{cite}

\usepackage[top=30truemm,bottom=30truemm,left=25truemm,right=25truemm]{geometry}
\setlength{\headheight}{15.5pt} % エラーで指定されている値よりも大きな値を設定する
\addtolength{\topmargin}{-3.5pt} % 必要に応じて topmargin も調整する


\begin{document}

\thispagestyle{empty}
\begin{center}

\vspace{20mm}
{\Large\noindent 2025年度 卒業論文}\\
\vspace{40mm}
{\huge\noindent\textbf{論文タイトル}}\\
\medskip
{\huge\noindent\textbf{論文タイトル(2行目)}}\\
\vspace{60mm}

%ここの空白文字は適当に
{\Large\noindent
2026年1月31日\\
\vspace{\baselineskip}
  所属 明治大学 \\
\vspace{\baselineskip}
指導教員 横山大作    \\
\vspace{\baselineskip}
 学籍番号 157R227127\\
\vspace{\baselineskip}
  名前 仲亜斗夢\\
}
\vspace{40mm}

\end{center}

\thispagestyle{empty}
\clearpage

%=====================================================================================
\renewcommand{\abstractname}{要旨}

\begin{abstract}
研究の要旨。なんやかんやなんやかんやなんやかんやなんやかんやなんやかんやなんやかんやなんやかんやなんやかんやなんやかんやなんやかんやなんやかんやなんやかんやなんやかんやなんやかんやなんやかんやなんやかんやなんやかんやなんやかんやなんやかんやなんやかんやなんなんや
\end{abstract}

%=====================================================================================

% 目次の表示
\tableofcontents

%=====================================================================================
\pagestyle{fancy}
\lhead{\rightmark}
\renewcommand{\chaptermark}[1]{\markboth{第\ \normalfont\thechapter\ 章~~#1}{}}
%=====================================================================================

\chapter{はじめに} %章


\section{研究背景} %1.1
\subsection{デジタルカーリングとは} %1.1.1
デジタルカーリングとは、コンピュータ上の物理シミュレータを用いた仮想的なカーリングスペースであり、カーリングの戦略を議論・検証するためのプラットフォームとして提案されたものである\cite{uec_cup_report}。伊藤らによる第1回UEC杯デジタルカーリング大会報告によると、本システムは二人零和有限確定完全情報ゲームとしての性質を持ちつつも、投球結果に確率的な誤差が含まれる「不確定ゲーム」として定義される。

\subsubsection*{システムの概要と物理演算}
デジタルカーリングでは、2次元物理演算エンジン「Box2D」を用いてストーンの挙動をシミュレートしてい。ストーンの軌道計算においては、プレイヤーが指定した初速度ベクトルと回転方向に対し,正規分布に従うランダムなノイズ(外乱)が加算される。これにより、現実のカーリングにおける「氷の状態変化」や「投球のブレ」に相当する不確定性が再現されている。

\subsubsection*{基本ルールと試合の進行}
試合の進行およびルールは、実際のカーリング競技に準拠している.
\begin{itemize}
    \item \textbf{試合構成}: 通常10エンド(または8エンド)で行われ、各エンドにおいて先攻・後攻の2チームが交互に8投ずつ、計16投のストーンを投げる。
    \item \textbf{得点計算}: 全投球終了後、ハウス(円)の中心(ティー)に最も近いストーンを持つチームが相手チームのNo.1ストーンよりも内側にある自チームのストーンの数だけ得点を得る。敗北したチームの得点は0点となる。
    \item \textbf{手番の決定}: 第1エンドはコイントス等で決定し、第2エンド以降は前のエンドで得点したチームが先攻となる。両チーム無得点(ブランクエンド)の場合は、手番は交代しない.
    \item \textbf{無効となるストーン}: ホッグラインを超えなかったストーン、バックラインを完全に超えたストーン、サイドラインに接触したストーンはプレイエリアから除外される。
\end{itemize}

これらの仕様により、デジタルカーリングは単なる物理シミュレーションにとどまらず、不確定性リスクを管理しながら最適な着手を選択する高度な戦略性が求められるゲーム環境となっている。


\section{研究目的}
近年、シミュレーション環境を用いた競技型AIの研究が盛んに行われており、特にデジタルカーリングのように行動空間が連続で構成されるゲームにおいては、高度な意思決定アルゴリズムが求められている。モンテカルロ木探索(Monte Carlo Tree Search, MCTS)は多くの分野で成功を収めているが、連続行動空間では探索の効率が著しく低下するという課題がある。特に大渡ら[1]の研究や加藤ら[2]の研究では、デジタルカーリングシステムにおけるゲーム木探索の有効性が示される一方で、候補手の絞り込みや、連続行動空間における最適な手の選択が依然として課題であることが明記されている。
一方、連続空間への対応としては深層強化学習も注目されているが、これらは学習コストが高く、設計も複雑であり、特定の環境に依存しやすいという欠点を持つ。従来のMCTSにおいても、グリッドによる固定的な離散化が試みられているが、局面ごとの柔軟性に欠け、探索精度を犠牲にしてしまう恐れがある。本研究はこれらの先行研究の課題を考慮したうえで、行動空間の離散化に加えてクラスタリングを導入し、局面に応じて動的に有効な候補手を抽出・選別することで、探索の深さと精度を両立する手法を提案する。これにより、限られた計算資源の中でも有効な手の選択が可能となり、デジタルカーリングAIの発展に寄与することを目的とする。

\section{本論文の構想}
近年、デジタルカーリングの分野ではMCTSを基盤としたAIプログラムが公式大会において高い競技成績を残しており、その有効性が実証されている。例えば、第10回UEC杯では大渡が開発したMCTSを用いた「歩」が優勝し[3]、さらに加藤らによって開発されたExpectimax-searchに基づく「じりつくん」シリーズは複数の大会で上位に入賞しており[4]、第11回大会では改良版である「Jiritsukun-Jr」が優勝を果たしている[5]。これらの成果は、MCTSがデジタルカーリングにおいて実践的かつ強力な手法であることを示している。
しかし、MCTSを連続行動空間であるデジタルカーリング環境[6]にそのまま適用するには課題が残されている。具体的には、無限に存在する候補手の中から探索対象を選定する必要があるため、計算資源が広く分散し、探索の効率が著しく低下してしまうという問題である。実際大渡らや加藤らの研究においても、候補手の絞り込みや行動の最適化が今後の課題として指摘されている。本研究では、連続行動空間を離散化し、さらにクラスタリングによって類似した行動を統合・代表化することで、効率よく有効な候補手を選定する手法を提案する。その有効性を検証するために、クラスタリングを導入したMCTS(以後、提案手法)と、クラスタリング無しの従来のMCTSとの比較実験を行う。評価は、①探索木の成長傾向、②候補手の選択分布、③対戦成績の3点から行う。
まず、各手法が生成する探索木を比較し、クラスタリングにより選ばれた代表候補手(集合K)のうち、実際に訪問回数の多かった手の割合を測定する。これにより、クラスタリングが有効な手を適切に抽出できたかを評価する。
次に、両手法を無制限時間下で対戦させ、その勝敗結果を性能指標とする。提案手法では、候補手集合Cの一部要素を代表手集合Kの要素に置き換えて、探索深さ3でプレイアウトを行い、選択された手の分布も記録する。
最後に、実際に両手法を対戦させ、提案手法が従来手法に比べて探索効率や精度の点で優れているかを評価し、その有効性を明らかにする。

%これまでに調べた大事な技法は章単位で書いて良い
%ex. モンテカルロ, NN, 色彩について

\chapter{関連研究}
%調べたことは第1章よりもこっちの方が良い。
%研究目的に必要な情報以外はこっちに書くイメージ
[1]	大渡勝己, 田中哲朗:カーリングAIに対するモンテカルロ木探索の適用, ゲームプログラミングワークショップ2016論文集, pp. 180-187(2016).
[2]	加藤修, 飯塚博幸, 山本雅人:不確定性を含むデジタルカーリングにおけるゲーム木探索, 情報処理学会論文誌ジャーナル, Vol.57, No.11, pp.2354-2364(2016).
[4]	伊藤毅志, 森健太郎:デジタルカーリング大会報告2015年度, 情報処理学会研究報告, 2016-GI-36, No.2, pp. 1 - 5(2016).
[8]	佐藤佳州, 高橋大介:モンテカルロ木探索によるコンピュータ将棋, 情報処理学会論文誌, Vol.50, No.11, pp. 2740-2751(2019).
%http://minerva.cs.uec.ac.jp/cgi-bin/curling/wiki.cgiデジタルカーリング大会報告2015年度, 情報処理学会研究報告
デジタルカーリングに関する研究は、ゲームの不確定性や連続空間という特性を扱うためにいくつかのアプローチが提案されている。

大渡ら[1]は、連続状態空間を持つデジタルカーリングに対して、モンテカルロ木探索(MCTS)を適用する手法を提案している。彼らの手法では、連続的な状態空間を階層的に分割する「状態木(State Tree)」という構造を導入することで、無限に存在する状態を有限のノード集合として扱い、MCTSによる探索を可能にした。実験の結果,単純なシミュレーション方策を用いた場合と比較して、提案手法が有効であることを示しており、連続空間における探索において適切な離散化や状態の抽象化が重要であることを示唆している。

一方、加藤ら[2]は不確定性を含むデジタルカーリングに対してExpectimax法によるゲーム木探索を提案している。彼らは投球目標座標と回転方向の組み合わせを候補手とし、実行時に加わる外乱(不確定性)を考慮して、確率的に遷移するチャンスノードをゲーム木に導入した。不確定性を考慮しない場合と考慮する場合の比較実験を行い、不確定性を考慮して先読みを行うことで、既存の強力なAIプログラムである「GCCS」に対しても高い勝率を達成できることを示した。

これらの先行研究は、デジタルカーリングにおける「連続空間の離散化」や「不確定性の考慮」が極めて重要であることを示している。本研究ではこれらの知見を踏まえつつ、候補手の生成においてクラスタリングを用いることで、より効率的に有効な手を探索する手法について議論する。

\chapter{作成した分析用ツール}
研究の過程で、シミュレーションの盤面状態を視覚的に確認するためのツールを作成した。デジタルカーリングの研究では、数値データ(ストーンの座標 $(x, y)$)だけでは戦況を直感的に把握することが困難である。そこで、PythonおよびMatplotlibを用いて、ストーン情報を入力するだけでカーリングシート上の配置を可視化するスクリプトを実装した。

本ツールは以下の機能を持つ。
\begin{itemize}
    \item \textbf{シートの描画}: 実際の競技ルールに基づいた正確な寸法で、ハウス(12ft, 8ft, 4ft, Button)、ホッグライン、ティーラインなどを描画する。
    \item \textbf{ストーンの配置}: 任意の個数のストーンについて、座標と所属チーム(先攻・後攻)を指定することで、シート上にプロットする。
    \item \textbf{チームの識別}: チーム0(先攻)を黄色、チーム1(後攻)を赤色で色分けし、視認性を高めている。
\end{itemize}

このツールを用いることで、MCTSが生成した「理想的なショット」や、実験で用いる「特定のテストケース(盤面)」がどのような状況であるかを即座に可視化・画像化することが可能となり、実験結果の分析やデバッグ効率が大幅に向上した。

%ここにシートに変換される様子が分かる画像を貼る
\begin{figure}[H]
 \begin{minipage}{0.5\textwidth}
  \centering
  \makeatletter\def\@captype{table}\makeatother
  \caption{ストーンの配置サンプルデータ}
  \begin{tabular}{|l|r|r|r|} \hline
   チーム & ストーンID & $x$座標 & $y$座標 \\ \hline
   0(先行) & 0 & 0.0 & 38.405 \\
   1(後攻) & 1 & -1.0 & 39.0 \\ 
   0(先行) & 2 & -1.0 & 37.810 \\ 
   1(後攻) & 3 & 1.0 & 38.405 \\ \hline
  \end{tabular}
 \end{minipage}
 \hfill
 \begin{minipage}{0.46\textwidth}
  \centering
  \includegraphics[width=\textwidth]{figure/data_to_stone_positions.png}
  \caption{表3.1の可視化結果}
 \end{minipage}
\end{figure}

\chapter{ルールベース+MCTSプレイヤーの作成}
本章では、提案手法の比較対象(ベースライン)として作成したルールベース戦略をモンテカルロ木探索(MCTS)に組み込んだプレイヤーについて述べる。

\section{ルールベースの規則}
本プレイヤーは、カーリングの定石に基づいた以下の5つの基本戦略(ルール)を実装しており、局面に応じてこれらを選択肢として生成する。

\begin{enumerate}
    \item \textbf{ガード戦略}: 
    ハウス手前にガードストーンを配置し、自チームのストーンを守るときや相手の攻撃ストーンの進路を遮る戦略。序盤や、相手が有利な状況で多用される。先行の場合はセンターガード、後攻の場合はコーナーガードを選択する。
    
    \item \textbf{重心戦略}: 
    ハウス内にある相手ストーン群の重心を計算し、その位置を狙う戦略。相手ストーンが散らばっている場合に、まとめて一掃したり、その裏に隠れたりするために用いる。盤面を意図的に崩した方が自チームが優位になる場合に使用される。
    
    \item \textbf{ドロー戦略}: 
    ハウスの中心(ボタン)を狙ってストーンを投げる、最も基本的な戦略。得点を確保するために重要であり、かつ、強い一手である。
    
    \item \textbf{テイクアウト戦略}: 
    ハウス内に存在する相手のNo.1ストーン(最も中心に近い石)を狙い、弾き出す戦略。相手の得点を防ぐために有効である。
    
    \item \textbf{高速球戦略}: 
    非常に速い速度($v_y \approx 3.0$ m/s)でストーンを投げ、ハウス内の石を乱したり、ガードを強引に弾き飛ばしたりする攻撃的な戦略。終盤でリスクを取って局面を打開する際に選択される。
\end{enumerate}

\section{MCTSのノードの選択方法}
本手法におけるノード展開では、前述の5つの戦略から候補手を選定する。すべての戦略を均等に探索するのではなく、エンドの進行状況(投球数)に応じて候補を絞り込むヒューリスティックを導入している。

\begin{itemize}
    \item \textbf{序盤 (ショット数 $< 5$)}: 
    先行の場合なら最初の3投、後攻の場合なら最初の2投の投球では、場を整えることが重要であるため、強制的に\textbf{ガード戦略}を選択する。このフェーズでは探索による勝利確率ではなく、ガードの位置評価関数(センターラインへの近さやカバー率)に基づくスコア(\texttt{EvaluateGuardStone})を用いてUCT値を計算する。
    
    \item \textbf{中盤以降}: 
    続いて6投目までは、攻撃の選択肢を増やす。
    \begin{itemize}
        \item ショット数 $ < 11$: \textbf{重心、ドロー、テイクアウト}からランダムに選択。
        \item ショット数 $\ge 11$: \textbf{高速球}を含めた全攻撃戦略からランダムに選択。
    \end{itemize}
\end{itemize}

また、探索の多様性を確保するため、親ノードで選択された戦略と同じ戦略を連続して選ばないように制約を加えている(\texttt{previous\_choice}による排他制御)。


\section{局面評価関数の実装}
MCTSのシミュレーションにおいて、各局面の良し悪しを判定するために、ヒューリスティックな評価関数(\texttt{EvaluateBoard})を実装した。この関数は、単にハウス内ストーンの得点計算を行うだけでなく、有利な状況を数値化して評価値 $S$ を算出する。具体的な評価項目は以下の通りである。

\begin{itemize}
    \item \textbf{距離スコア}: ハウス内にある自チームの各ストーンについて、中心からの距離に応じたスコアを加算する。
    \[ S_{dist} = \sum_{s \in \text{MyStones}} \max(0, 4.0 - \text{dist}(s, \text{Center})) \]
    \item \textbf{ストーン数優位}: ハウス内の有効ストーン数が相手より多い場合、加点($+3.0$)する。
    \item \textbf{ガードストーン評価}: ハウス手前に配置されたガードストーンについて、自チームのものは加点($+2.0$)、相手チームのものは減点($-2.0$)とする。
    \item \textbf{No.1ストーンボーナス}: 自チームのストーンがハウス中心に最も近い(No.1ストーンである)場合、大きなボーナス点($+20.0$)を加算する。
    \item \textbf{弾き出しボーナス}: 直前の状態と比較して、相手ストーンを弾き出した(ハウス外に出した)場合や、自チームのストーンが増えた場合にボーナスを加算する。
\end{itemize}

最終的な評価値はこれらの合計として算出され、MCTSはこの値に基づいて有利な分岐を探索する。

\section{対戦結果}
本手法を用いて、電気通信大学が主催する第10回GAT (Game AI Tournaments) デジタルカーリング部門に参加した。その結果、参加10チーム中9位という成績に終わった。
以下に、この結果に至った敗因と、大会を通じて得られた課題について述べる。

\subsection{探索不足による精度の低下}
最大の課題は、思考時間の制約によりMCTSの十分な探索数を確保できなかった点である。デジタルカーリングは連続空間におけるゲームであり、正確なショットを選択するためには膨大な数のシミュレーションが不可欠である。しかし、実時間制約(1手あたりの制限時間)を守るために探索回数(Iteration)を制限せざるを得ず、その結果、読みの深さが不足し、相手の戦略に対する有効な対抗手を発見できないケースが多発した。

\subsection{ルールベースによる攻撃の硬直化}
また、候補手生成をルールベースに依存していたことも敗因の一つである。本手法では、あらかじめ定義された5つの戦略(ガード、ドロー、テイクアウト等)の中からしか手を選べないため、盤面の微妙な変化に対応した柔軟なショット(例えば、少しずらして複数の石に当てるウィックや、複雑なコンビネーションショット等)を生成することができなかった。
これにより、攻撃のバリエーションが限定され、相手に戦略を読まれやすい単調なプレイ展開となってしまった。これらの経験から、ルールベースの候補手生成だけでは限界があり、より多様な候補手を探索空間から自律的に抽出する仕組みが必要であることを認識した。


\chapter{クラスタリング+MCTSプレイヤーの提案}
\section{クラスタリングの目的}

\section{実装方法}

\chapter{クラスタリング+MCTSプレイヤーの評価}
\section{クラスタリングの評価方法}

\section{評価関数の定義}

\section{AllGrid v.s. Clusteredの対比実験}

\section{実験結果}


\chapter{まとめ}
研究のまとめ。なんやかんやなんやかんやなんやかんやなんやかんやなんやかんやなんやかんやなんやかんやなんやかんやなんやかんやなんやかんやなんやかんやなんやかんやなんやかんやなんやかんやなんやかんやなんやかんやなんやかんやなんやかんやなんやかんやなんやかんや

%=====================================================================================
% \chapter*{謝辞} %章を付けずにタイトル表示
% \addcontentsline{toc}{chapter}{謝辞} %章立てせずに目次に追加するおまじない
% 本論文を作成するにあたり、---- みなさまに感謝の意を表します.


%=====================================================================================

\addcontentsline{toc}{chapter}{参考文献} %章立てせずに目次に追加するおまじない
\renewcommand{\bibname}{参考文献} %これがないと,タイトルが「関連図書」になってしまう
\bibliographystyle{junsrt} %本文に\cite{}を入れることで,参考文献表示
\bibliography{refer} %bibtexファイルの読み込み


\end{document}

