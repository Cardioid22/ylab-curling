\documentclass[11pt,a4j]{jreport}

\usepackage{comment}
\usepackage{float}
\usepackage{color}
\usepackage{multicol}
\usepackage[dvipdfmx]{pict2e}
\usepackage{wrapfig}
\usepackage{graphicx}
\usepackage{bm}
\usepackage{url}
\usepackage{underscore}
\usepackage{colortbl}
\usepackage{tabularx}
\usepackage{fancyhdr}
\usepackage{ulem}
\usepackage{amsmath,amssymb,amsfonts}
\usepackage{algorithmic}
\usepackage{textcomp}
\usepackage{xcolor}
\usepackage[ipaex]{pxchfon}
\usepackage{algorithmic}
\usepackage{algorithm}
\usepackage{cite}

\usepackage[top=30truemm,bottom=30truemm,left=25truemm,right=25truemm]{geometry}
\setlength{\headheight}{15.5pt} % エラーで指定されている値よりも大きな値を設定する
\addtolength{\topmargin}{-3.5pt} % 必要に応じて topmargin も調整する


\begin{document}

\thispagestyle{empty}
\begin{center}

\vspace{20mm}
{\Large\noindent 2025年度 卒業論文}\\
\vspace{40mm}
{\huge\noindent\textbf{連続行動空間におけるMCTSでのクラスタリング導入による探索効率化手法の提案
}}\\
\medskip
{\huge\noindent\textbf{論文タイトル(2行目)}}\\
\vspace{60mm}

%ここの空白文字は適当に
{\Large\noindent
2026年1月31日\\
\vspace{\baselineskip}
  所属 明治大学 \\
\vspace{\baselineskip}
指導教員 横山大作    \\
\vspace{\baselineskip}
 学籍番号 157R227127\\
\vspace{\baselineskip}
  名前 仲亜斗夢\\
}
\vspace{40mm}

\end{center}

\thispagestyle{empty}
\clearpage

%=====================================================================================
\renewcommand{\abstractname}{要旨}

\begin{abstract}
デジタルカーリングは、コンピュータ上の物理シミュレータを用いた仮想的なカーリング環境であり、連続行動空間と確率的な不確定性を併せ持つ高度な戦略ゲームである。モンテカルロ木探索(MCTS)はこのようなゲームにおいて有効な手法として知られているが、連続行動空間では候補手が無限に存在するため、探索効率が著しく低下するという課題がある。

本研究では、この課題に対処するため、行動空間の離散化に加えてクラスタリングを導入し、局面に応じて動的に有効な候補手を抽出・選別する手法を提案する。具体的には、まずハウス周辺の座標をグリッド化して候補手を生成し、各候補手のシミュレーション結果に対して階層的クラスタリングを適用することで、類似した結果をもたらす候補手を統合する。これにより、MCTSの子ノード数を大幅に削減し、探索効率の向上を図る。

さらに、クラスタリングによって失われる微細な座標情報を補うため、「Zoom-In探索」を導入する。これは、MCTSによって最も有望と判断されたクラスタ内の候補手に対して集中的に再探索を行うことで、最終的な着手精度を向上させるアプローチである。

評価実験では、10種類のテストジャンルから生成した100ケースの局面を用いて、提案手法(Clustered MCTS)と全探索手法(AllGrid MCTS)を比較した。その結果、提案手法は全探索の10\%〜30\%程度の計算量で約80\%のクラスタ一致率を達成した。一方、完全一致率は約20\%に留まり、クラスタリングのみでは最適解の特定に限界があることも確認された。この結果は、Zoom-In探索による精密化の必要性を裏付けるものである。
\end{abstract}

%=====================================================================================

% 目次の表示
\tableofcontents

%=====================================================================================
\pagestyle{fancy}
\lhead{\rightmark}
\renewcommand{\chaptermark}[1]{\markboth{第\ \normalfont\thechapter\ 章~~#1}{}}
%=====================================================================================

\chapter{はじめに}

\section{研究背景}
近年、シミュレーション環境を用いた競技型AIの研究が盛んに行われており、特にデジタルカーリングのように行動空間が連続で構成されるゲームにおいては、高度な意思決定アルゴリズムが求められている。モンテカルロ木探索(Monte Carlo Tree Search, MCTS)は多くの分野で成功を収めているが、連続行動空間では探索の効率が著しく低下するという課題がある。特に大渡らの研究や加藤らの研究では、デジタルカーリングシステムにおけるゲーム木探索の有効性が示される一方で、候補手の絞り込みや、連続行動空間における最適な手の選択が依然として課題であることが明記されている。
従来のMCTSにおいても、グリッドによる固定的な離散化が試みられているが、局面ごとの柔軟性に欠け、探索精度を犠牲にしてしまう恐れがある。

\section{研究目的}
本研究はこれらの先行研究の課題を考慮したうえで、行動空間の離散化に加えてクラスタリングを導入し、局面に応じて動的に有効な候補手を抽出・選別することで、探索の深さと精度を両立する手法を提案する。これにより、限られた計算資源の中でも有効な手の選択が可能となり、デジタルカーリングAIの発展に寄与することを目的とする。

\section{研究貢献}
本研究の主な貢献は以下の通りである。
\begin{enumerate}
    \item \textbf{クラスタリングによる候補手の効率的な絞り込み手法の提案}: 連続行動空間を離散化した後、クラスタリングによって類似した行動を統合・代表化することで、探索空間を大幅に削減しつつ、有効な候補手を維持する手法を提案した。
    \item \textbf{Zoom-In探索による精密化手法の導入}: クラスタリングで失われる微細な情報を補うため、有望クラスタ内で再探索を行うZoom-In探索を導入し、探索精度の向上を実現した。
    \item \textbf{提案手法の定量的評価}: 全探索手法との比較実験を行い、クラスタ一致率および完全一致率の観点から提案手法の有効性を定量的に示した。
\end{enumerate}

\section{研究成果}
本研究の実験により、以下の成果が得られた。
\begin{itemize}
    \item クラスタリングを用いた探索では、全探索の10\%〜30\%程度の計算量で、約60\%のクラスタ一致率を達成した。
    \item 一方、完全一致率は約20\%に留まり、クラスタリングのみでは微細な座標の特定に限界があることが確認された。
    \item Zoom-In探索の導入により、有望クラスタ内での集中探索が可能となり、最終的な候補手選択精度の向上が期待される。
\end{itemize}

\section{本論文の構成}
本論文は以下の構成で記述される。
第2章では、デジタルカーリングの概要と関連研究について述べる。
第3章では、研究過程で作成した分析用ツールについて説明する。
第4章では、デジタルカーリング大会出場のために最初に作成したルールベース+MCTSプレイヤーについて述べる。
第5章では、本研究の提案手法であるクラスタリングを用いた探索手法およびZoom-In探索について詳述する。また、比較対象となるベースライン手法(AllGrid MCTS:グリッド上の全候補手を探索する手法)についても説明し、提案手法との比較実験結果を示す。
第6章では、本研究のまとめと今後の課題について述べる。

%これまでに調べた大事な技法は章単位で書いて良い
%ex. モンテカルロ, NN, 色彩について


\chapter{デジタルカーリングと関連研究}

\section{デジタルカーリングとは}
デジタルカーリングとは、コンピュータ上の物理シミュレータを用いた仮想的なカーリングスペースであり、カーリングの戦略を議論・検証するためのプラットフォームとして提案されたものである\cite{digitalcurling_server, uec_cup_report}。北清らによってデジタルカーリングサーバーが提案され、伊藤らによる第1回UEC杯デジタルカーリング大会報告によると、本システムは二人零和有限確定完全情報ゲームとしての性質を持ちつつも、投球結果に確率的な誤差が含まれる「不確定ゲーム」として定義される\cite{kato2016}。

\subsection{システムの概要と物理演算}
デジタルカーリングでは、2次元物理演算エンジン「Box2D」を用いてストーンの挙動をシミュレートしている\cite{digitalcurling_server}。ストーンの軌道計算においては、プレイヤーが指定した初速度ベクトルと回転方向に対し、正規分布に従うランダムなノイズ(外乱)が加算される。これにより、現実のカーリングにおける「氷の状態変化」や「投球のブレ」に相当する不確定性が再現されている\cite{kato2016}。

\subsection{基本ルールと試合の進行}
試合の進行およびルールは、実際のカーリング競技に準拠している\cite{digitalcurling_server, kato2016}。
\begin{itemize}
    \item \textbf{試合構成}: 通常10エンド(または8エンド)で行われ、各エンドにおいて先攻・後攻の2チームが交互に8投ずつ、計16投のストーンを投げる。
    \item \textbf{得点計算}: 全投球終了後、ハウス(円)の中心(ティー)に最も近いストーンを持つチームが相手チームのNo.1ストーンよりも内側にある自チームのストーンの数だけ得点を得る。敗北したチームの得点は0点となる。
    \item \textbf{手番の決定}: 第1エンドはコイントス等で決定し、第2エンド以降は前のエンドで得点したチームが先攻となる。両チーム無得点(ブランクエンド)の場合は、手番は交代しない。
    \item \textbf{無効となるストーン}: ホッグラインを超えなかったストーン、バックラインを完全に超えたストーン、サイドラインに接触したストーンはプレイエリアから除外される。
\end{itemize}

これらの仕様により、デジタルカーリングは単なる物理シミュレーションにとどまらず、不確定性リスクを管理しながら最適な着手を選択する高度な戦略性が求められるゲーム環境となっている。

\section{関連研究}
デジタルカーリングに関する研究は、ゲームの不確定性や連続空間という特性を扱うためにいくつかのアプローチが提案されている。本節では、まず本研究で用いるモンテカルロ木探索(MCTS)の基礎的な手法について説明し、その後デジタルカーリングへの適用に関する先行研究を概観する。

\subsection{モンテカルロ木探索(MCTS)とUCT}
モンテカルロ木探索(Monte Carlo Tree Search, MCTS)は、探索木とシミュレーションを用いて最善手を探索する手法であり、特にゲームAI分野で広く用いられている\cite{imagawa2018mcts}。MCTSの基本的な考え方は、現在の局面から可能な行動を展開し、ランダムシミュレーションによって評価を行い、良い結果を出した手を重点的に探索するというものである。

MCTSは以下の4つのステップを繰り返すことで探索木を成長させる。
\begin{enumerate}
    \item \textbf{選択(Selection)}: 既存ノードの中から、後述するUCB値などの基準に基づいて最も有望なノードを選択する。
    \item \textbf{展開(Expansion)}: 選択されたノードから未探索のノードを追加する。
    \item \textbf{評価(Simulation / Playout)}: 追加されたノードからランダムに終局まで進めて(プレイアウト)、その結果を評価する。
    \item \textbf{更新(Backpropagation)}: シミュレーションの結果を、選択された経路上の親ノードへ遡って反映する。
\end{enumerate}

MCTSの代表的なアルゴリズムとして、UCT(Upper Confidence Bound applied to Trees)がある。UCTは多腕バンディット問題(Multi-Armed Bandit Problem)の理論を応用したものであり、各ノードの選択にUCB(Upper Confidence Bound)値を用いる。UCB値は以下の式で計算される。

\begin{equation}
    UCB_j = \bar{X}_j + C_p \sqrt{\frac{\ln n}{N_j}}
\end{equation}

ここで、$\bar{X}_j$はノード$j$における平均報酬、$N_j$はノード$j$の訪問回数、$n$は親ノードの総訪問回数、$C_p$は探索と活用のバランスを調整する探索係数である。第1項は活用(exploitation)を、第2項は探索(exploration)を表しており、訪問回数が少ないノードほど第2項が大きくなることで、未探索のノードも適切に選択される仕組みとなっている。

MCTSの特徴として、探索を途中で打ち切っても現時点での最善手を出力できる「Anytime性」、十分な探索回数を与えれば理論的に最適解に収束する「漸近的最適性」、そして探索性能が計算資源(時間・メモリ)に依存する「計算資源依存性」が挙げられる。


\subsection{デジタルカーリングとモンテカルロ木探索}
大渡ら\cite{owatari2016}は、連続状態空間を持つデジタルカーリングに対して、モンテカルロ木探索(MCTS)を適用する手法を提案している。デジタルカーリングは状態空間と行動空間がともに連続であり、さらに行動決定後に加えられる外乱の影響によって行動後の状態が一意に定まらないという、現実世界での意思決定における困難な問題を内包している。

彼らの手法では、連続的な状態空間を階層的に分割する「状態木(State Tree)」という構造を導入することで、無限に存在する状態を有限のノード集合として扱い、MCTSによる探索を可能にした。状態木の各ノードはそれぞれ範囲を持った状態集合を表し、木が深くなるにつれてより細かい状態の区別が可能となる。また、近い空間においては行動価値も近いという性質を利用し、似た状態のシミュレーション結果を用いて新奇な状態における行動価値を近似する手法を考案した。

行動空間の離散化については、加藤らの手法を参考に目標到達座標を離散化している。具体的には、ドロー系のショットとしてハウス内とその前方に等間隔で目標到達座標の点を配置し(63×31個)、テイクアウト系のショットとして遠方に一列の目標到達座標の点を配置している(189×1個)。回転方向が左右の2種類あるため、候補ショット数は合計で4284個となっている。

実験の結果、単純なシミュレーション方策を用いた場合だけでなく、カーリングの知識を用いた複雑なシミュレーション方策を用いた場合にも提案手法が有効であることを確認した。しかし、今後の課題として「連続な行動空間において行動を最適化すること」および「状態木をより効率良く分割すること」が挙げられており、探索空間の効率的な削減が依然として重要な研究課題であることを示唆している。

\subsection{不確定性を考慮したゲーム木探索}
加藤ら\cite{kato2016}は、不確定性を含むデジタルカーリングに対してExpectimax法によるゲーム木探索を提案している。Expectimaxとは、通常のMinimax法に対しチャンスノードと呼ばれる確率的な分岐点を導入し、不確定ゲームにおいてMinimax探索を行う手法である。

デジタルカーリングへのゲーム木探索の適用にあたっては、2つの解決すべき課題がある。1つ目は、候補手である投球目標座標が二次元直交座標系の実数値ベクトルであり、候補手が無限に存在することである。2つ目は、プレイヤが入力するストーンの初速度に実数値の乱数が加えられることで、1つの候補手から生成されうる結果の局面が毎回異なり、想定するノードが無限に存在することである。

これらの課題に対処するため、彼らは投球目標座標を一定範囲内の領域で等間隔に離散化した投球目標座標集合を生成している。具体的には、プレイエリア周辺を投球目標座標とする集合$Q_p$(主にドロー系、座標間隔はストーン半径$r_s$で1593個)と、プレイエリア遠方を投球目標座標とする集合$Q_f$(主にテイクアウト系、座標間隔は$0.5r_s$で72個)を定義し、回転方向2種類を考慮して合計4284個の候補ショットを生成している。

また、乱数による不確定性への対処として、候補手から生成されうる実行手を有限個の「代表実行手集合」として定義し、各実行手への分岐確率を事前に計算した確率テーブルを用いてチャンスノードの評価値を算出している。

実験では、不確定性を考慮する場合としない場合それぞれにおいて探索の深さを変化させ、既存AIとの対戦実験を行った。その結果、不確定性を考慮した場合に勝率が上昇し、また不確定性を考慮した場合のみ探索の深さを増やすことで勝率が上昇した。このことから、デジタルカーリングにおける不確定性を考慮した先読みの有効性が明らかとなった。さらに、今後の課題として「候補手の絞り込みを行うことでさらに深い探索を行い、不確定ゲームにおいて深く探索することの意義をより明らかにする」ことが述べられている。

\subsection{先行研究の課題と本研究の位置づけ}
これらの先行研究に共通する特徴として、連続行動空間をグリッドによって離散化し、有限個の候補手として扱うアプローチが採用されている点が挙げられる。大渡らは4284個、加藤らも同様に4284個の候補ショットを生成しており、これらはいずれもハウス周辺の座標を等間隔で区切ったグリッドに基づいている。

しかし、両研究ともに今後の課題として候補手の絞り込みや探索空間の効率的な削減を挙げている。大渡らは「連続な行動空間において行動を最適化すること」を、加藤らは「候補手の絞り込みを行うことでさらに深い探索を行う」ことを課題として明記している。これは、グリッドによる離散化だけでは探索空間が依然として大きく、限られた計算資源の中で十分な深さの探索を行うことが困難であることを示している。

本研究では、これらの先行研究の知見と課題を踏まえ、グリッドによる離散化に加えてクラスタリングを導入することで、局面に応じて動的に有効な候補手を抽出・選別する手法を提案する。具体的には、各候補手のシミュレーション結果に対して階層的クラスタリングを適用し、類似した結果をもたらす候補手を統合することで、MCTSの子ノード数を大幅に削減する。これにより、先行研究で課題とされていた「候補手の絞り込み」を実現し、限られた計算資源の中でもより効率的な探索を可能にすることを目指す。

\chapter{作成した分析用ツール}
研究の過程で、シミュレーションの局面状態を視覚的に確認するためのツールを作成した。デジタルカーリングの研究では、数値データ(ストーンの座標 $(x, y)$)だけでは戦況を直感的に把握することが困難である。そこで、PythonおよびMatplotlibを用いて、ストーン情報を入力するだけでカーリングシート上の配置を可視化するスクリプトを実装した。

本ツールは以下の機能を持つ。
\begin{itemize}
    \item \textbf{シートの描画}: 実際の競技ルールに基づいた正確な寸法で、ハウス(12ft, 8ft, 4ft, Button)、ホッグライン、ティーラインなどを描画する。
    \item \textbf{ストーンの配置}: 任意の個数のストーンについて、座標と所属チーム(先攻・後攻)を指定することで、シート上にプロットする。
    \item \textbf{チームの識別}: チーム0(先攻)を黄色、チーム1(後攻)を赤色で色分けし、視認性を高めている。
\end{itemize}

このツールを用いることで、MCTSが生成した「理想的なショット」や、実験で用いる「特定のテストケース(局面)」がどのような状況であるかを即座に可視化・画像化することが可能となり、実験結果の分析やデバッグ効率が大幅に向上した。

%ここにシートに変換される様子が分かる画像を貼る
\begin{figure}[H]
 \begin{minipage}{0.5\textwidth}
  \centering
  \makeatletter\def\@captype{table}\makeatother
  \caption{ストーンの配置サンプルデータ}
  \begin{tabular}{|l|r|r|r|} \hline
   チーム & ストーンID & $x$座標 & $y$座標 \\ \hline
   0(先行) & 0 & 0.0 & 38.405 \\
   1(後攻) & 1 & -1.0 & 39.0 \\ 
   0(先行) & 2 & -1.0 & 37.810 \\ 
   1(後攻) & 3 & 1.0 & 38.405 \\ \hline
  \end{tabular}
 \end{minipage}
 \hfill
 \begin{minipage}{0.46\textwidth}
  \centering
  \includegraphics[width=\textwidth]{figure/data_to_stone_positions.png}
  \caption{表3.1の可視化結果}
 \end{minipage}
\end{figure}

\chapter{ルールベース+MCTSプレイヤーの作成}
本章では、提案手法の比較対象(ベースライン)として作成したルールベース戦略をモンテカルロ木探索(MCTS)に組み込んだプレイヤーについて述べる。

\section{ルールベースの規則}
本プレイヤーは、カーリングの定石に基づいた以下の5つの基本戦略(ルール)を実装しており、局面に応じてこれらを選択肢として生成する。

\begin{enumerate}
    \item \textbf{ガード戦略}: 
    ハウス手前にガードストーンを配置し、自チームのストーンを守るときや相手の攻撃ストーンの進路を遮る戦略。序盤や、相手が有利な状況で多用される。先行の場合はセンターガード、後攻の場合はコーナーガードを選択する。
    
    \item \textbf{重心戦略}: 
    ハウス内にある相手ストーン群の重心を計算し、その位置を狙う戦略。相手ストーンが散らばっている場合に、まとめて一掃したり、その裏に隠れたりするために用いる。局面を意図的に崩した方が自チームが優位になる場合に使用される。
    
    \item \textbf{ドロー戦略}: 
    ハウスの中心(ボタン)を狙ってストーンを投げる、最も基本的な戦略。得点を確保するために重要であり、かつ、強い一手である。
    
    \item \textbf{テイクアウト戦略}: 
    ハウス内に存在する相手のNo.1ストーン(最も中心に近い石)を狙い、弾き出す戦略。相手の得点を防ぐために有効である。
    
    \item \textbf{高速球戦略}: 
    非常に速い速度($v_y \approx 3.0$ m/s)でストーンを投げ、ハウス内の石を乱したり、ガードを強引に弾き飛ばしたりする攻撃的な戦略。終盤でリスクを取って局面を打開する際に選択される。
\end{enumerate}

\section{MCTSのノードの選択方法}
本手法におけるノード展開では、前述の5つの戦略から候補手を選定する。すべての戦略を均等に探索するのではなく、エンドの進行状況(投球数)に応じて候補を絞り込むヒューリスティックを導入している。

\begin{itemize}
    \item \textbf{序盤 (ショット数 $< 5$)}: 
    先行の場合なら最初の3投、後攻の場合なら最初の2投の投球では、場を整えることが重要であるため、強制的に\textbf{ガード戦略}を選択する。このフェーズでは探索による勝利確率ではなく、ガードの位置評価関数(センターラインへの近さやカバー率)に基づくスコア(\texttt{EvaluateGuardStone})を用いてUCT値を計算する。
    
    \item \textbf{中盤以降}: 
    続いて6投目までは、攻撃の選択肢を増やす。
    \begin{itemize}
        \item ショット数 $ < 11$: \textbf{重心、ドロー、テイクアウト}からランダムに選択。
        \item ショット数 $\ge 11$: \textbf{高速球}を含めた全攻撃戦略からランダムに選択。
    \end{itemize}
\end{itemize}

また、探索の多様性を確保するため、親ノードで選択された戦略と同じ戦略を連続して選ばないように制約を加えている(\texttt{previous\_choice}による排他制御)。


\section{局面評価関数の実装}
MCTSのシミュレーションにおいて、各局面の良し悪しを判定するために、ヒューリスティックな評価関数(\texttt{EvaluateBoard})を実装した。この関数は、単にハウス内ストーンの得点計算を行うだけでなく、有利な状況を数値化して評価値 $S$ を算出する。具体的な評価項目は以下の通りである。

\begin{itemize}
    \item \textbf{距離スコア}: ハウス内にある自チームの各ストーンについて、中心からの距離に応じたスコアを加算する。
    \[ S_{dist} = \sum_{s \in \text{MyStones}} \max(0, 4.0 - \text{dist}(s, \text{Center})) \]
    \item \textbf{ストーン数優位}: ハウス内の有効ストーン数が相手より多い場合、加点($+3.0$)する。
    \item \textbf{ガードストーン評価}: ハウス手前に配置されたガードストーンについて、自チームのものは加点($+2.0$)、相手チームのものは減点($-2.0$)とする。
    \item \textbf{No.1ストーンボーナス}: 自チームのストーンがハウス中心に最も近い(No.1ストーンである)場合、大きなボーナス点($+20.0$)を加算する。
    \item \textbf{弾き出しボーナス}: 直前の状態と比較して、相手ストーンを弾き出した(ハウス外に出した)場合や、自チームのストーンが増えた場合にボーナスを加算する。
\end{itemize}

最終的な評価値はこれらの合計として算出され、MCTSはこの値に基づいて有利な分岐を探索する。

\section{対戦結果}
本手法を用いて、電気通信大学が主催する第10回GAT (Game AI Tournaments) デジタルカーリング部門に参加した。その結果、参加10チーム中9位という成績に終わった。
以下に、この結果に至った敗因と、大会を通じて得られた課題について述べる。

\subsection{探索不足による精度の低下}
最大の課題は、思考時間の制約によりMCTSの十分な探索数を確保できなかった点である。デジタルカーリングは連続空間におけるゲームであり、正確なショットを選択するためには膨大な数のシミュレーションが不可欠である。しかし、実時間制約(1手あたりの制限時間)を守るために探索回数(Iteration)を制限せざるを得ず、その結果、読みの深さが不足し、相手の戦略に対する有効な対抗手を発見できないケースが多発した。

\subsection{ルールベースによる攻撃の硬直化}
また、候補手生成をルールベースに依存していたことも敗因の一つである。本手法では、あらかじめ定義された5つの戦略(ガード、ドロー、テイクアウト等)の中からしか手を選べないため、局面の微妙な変化に対応した柔軟なショット(例えば、少しずらして複数の石に当てるショットや、複雑なコンビネーションショット等)を生成することができなかった。
これにより、攻撃のバリエーションが限定され、相手に戦略を読まれやすい単調なプレイ展開となってしまった。これらの経験から、ルールベースの候補手生成だけでは限界があり、より多様な候補手を探索空間から自律的に抽出する仕組みが必要であることを認識した。


\chapter{クラスタリングを用いた探索手法の提案と評価}
\section{クラスタリングの目的}
連続解空間において候補手を選定することは依然として課題となっている。それらの候補手を有限の範囲に絞って全探索することは大量の計算リソースを必要とする。それを改善する方法として本研究ではクラスタリングを用いる。クラスタリングによって大量にある候補手から類似手をまとめることで、探索空間を効率的に削減することを目的とする。

\section{ベースライン手法:AllGrid MCTS}
本研究では、提案手法の比較対象として「AllGrid MCTS」をベースライン手法として位置づける。

AllGrid MCTSとは、連続行動空間を離散化するためにハウス周辺の座標を$M \times N$のグリッドに分割し、そのグリッド上の全ての着手点を候補手としてMCTSの子ノードに展開する手法である。この手法は探索空間を間引くことなく網羅的に探索するため、十分な計算時間を与えれば、グリッド化された条件下における理論上の最善手に収束すると考えられる。

しかし、AllGrid MCTSには以下の課題がある。
\begin{itemize}
    \item \textbf{計算量の増大}: グリッドサイズが$M \times N$の場合、各ノードで$M \times N$個の子ノードを探索する必要があり、探索木の幅が非常に大きくなる。例えば$16 \times 16 = 256$通りの候補手がある場合、限られた計算時間内では各候補手に対する十分な探索回数を確保することが困難となる。
    \item \textbf{類似した候補手への重複投資}: 隣接するグリッド点は結果として類似した局面を生み出すことが多い。しかしAllGrid MCTSではこれらを別々の候補手として扱うため、戦略的に同等な選択肢に対して計算資源が分散してしまう。
\end{itemize}

本研究では、AllGrid MCTSによって十分な探索回数を確保した結果を「基準解(正解)」として扱い、提案手法(Clustered MCTS)がより少ない計算量でこの基準解にどの程度近づけるかを評価の指標とする。

\section{提案手法:Clustered MCTSの実装方法}
本研究で提案するクラスタリングを用いた候補手生成およびMCTSへの適用手法について、以下の5つの手順で説明する。

\begin{enumerate}
    \item \textbf{ハウス周辺の座標を離散化する(グリッド化)} \\
    まず、ハウスの中心周辺の領域を$M \times N$のグリッドに分割し、離散的な着手点を作成する。これにより、本来は連続的で無限に存在する着手位置を、有限個の代表的なターゲット位置の集合として扱うことが可能となる。

    \item \textbf{グリッドの座標に対して、試し投げ後の局面を記録する} \\
    次に、グリッドの各交点(ターゲット位置)に向けて、事前に算出した適切な初速度ベクトルを用いてシミュレーション(試し投げ)を行う。このシミュレーションはデジタルカーリングが提供する環境に基づいて行う。この結果得られた遷移後の局面状態(全ストーンの位置情報)を、各ターゲット位置に対応する「投球後の予測状態」としてすべて記録する。

    \item \textbf{投げ終わった後の局面の状態でグループ化を行う} \\
    記録された多数の予測状態に対して、局面の類似性に基づくクラスタリングを行う。本手法では、局面の特徴を適切に捉えるために、以下の特徴抽出と2段階のクラスタリングプロセスを採用している。

    まず、各局面から「特徴ベクトル」を抽出する。特徴量としては、局面上の総ストーン数、ハウス周辺を6つの領域(左上・中上・右上・左下・中下・右下)に分割した際の各領域における自チーム・相手チームのストーン数、ハウス内のストーン数、およびNo.1ストーン(最も中心に近い石)の所属チーム情報を用いる。

    クラスタリングは以下の手順で実行される。
    \begin{enumerate}
      \item \textbf{粗分類}: 局面上の「総ストーン数」が一致するもの同士でグループ分けを行う。これにより、シミュレーション結果としてストーン数が変化したもの(テイクアウト成功時など)とそうでないものを効率的に分離する。
      \item \textbf{詳細分類}: 各グループ内で、k-means法を用いてさらに細分化する。ここでの類似度(距離)計算には、単純なユークリッド距離ではなく、領域ごとのストーン分布の差、ハウス内のストーン数の差、No.1ストーンの不一致度などを重み付けして加算した指標を用いる。これにより、戦況の構造(有利不利の状況や石の配置バランス)が似ている局面同士を同一のクラスタにまとめ上げる。
    \end{enumerate}

    \item \textbf{各クラスタの代表点を選出し、MCTSの子ノードとする} \\
    生成された各クラスタの中から、そのクラスタを代表する一つの状態を選出する。クラスタ内に実在する状態の中から選出する。
    選出基準としては、静的な局面評価関数を用いて各状態のスコアを算出し、自チームにとって最も評価値(有利度)が高い状態を代表点として採用する。この代表状態に対応するショット(初速度ベクトル)が次の探索ステップにおける有力な候補手となる。

    \item \textbf{MCTSの探索結果より最善手を選出する} \\
    選出された各クラスタの代表手を、MCTSのルートノード(自チームにおける最新の状態)直下の子ノードとして展開する。これにより、グリッド化によって生成された大量(数千通り)の候補手を、クラスタ数と同程度(数個〜数十個)の「有望な手のパターン」にまで絞り込むことができる。
    MCTSはこの絞り込まれた代表ノードに対して重点的にシミュレーションを行い、UCBなどの基準に従って探索木を成長させる。最終的に、最も訪問回数が多く、勝率が高いと判断された代表手(クラスタ)を特定し、後述するZoom-In探索へと移行する。
\end{enumerate}

    \begin{figure}[H]
        \centering
        \begin{minipage}{0.48\textwidth}
            \centering
            \includegraphics[width=0.9\linewidth]{figure/shot_grid_visualization.png}
            \caption{着手点のグリッド化 ($M=4, N=4$)}
            \label{fig:shot_grid}
        \end{minipage}
        \hfill
        \begin{minipage}{0.48\textwidth}
            \centering
            \includegraphics[width=0.9\linewidth]{figure/clustering_regions.png}
            \caption{クラスタリングにおける6つの領域分割}
            \label{fig:clustering_regions}
        \end{minipage}
    \end{figure}

\section{クラスタリングの評価方法}
本節では、提案するクラスタリング手法の有効性を検証するための実験について述べる。「AllGrid MCTS」と「Clustered MCTS」の比較実験を行った。
AllGrid MCTSは、グリッド上の全ての着手(候補手)を子ノードとして展開する探索木を構築する手法である。これは、探索空間を間引くことなく網羅的に探索するため、十分な計算時間を与えれば、グリッド化された条件下における理論上の最善手に収束すると考えられる。本実験では、このAllGrid MCTSによって導き出された解を「正解」とみなす。

一方、Clustered MCTSは、前節で述べた通り、全グリッド状態をクラスタリングし、各クラスタの代表手のみを子ノードとして展開する手法である。この手法は、探索の幅を劇的に削減できるため、限られた計算資源での探索効率向上が期待される。

実験では、同一の局面に対して両手法を実行し、それぞれが選択した最終的な候補手(ベストショット)がどの程度一致するかを比較することで、クラスタリングによる代表手選出の妥当性を評価する。

\subsection{評価指標の定義}
一致率の評価として、本研究では以下の2つの指標を定義する。

\begin{itemize}

    \item \textbf{完全一致率(Exact Agreement Rate)}: \\
    AllGrid MCTSが選択した候補手と、Clustered MCTSが選択した候補手が全く同一のグリッドID(ターゲット座標)である割合。これが高いほど、クラスタリングを行っても情報の損失がなく、本来の最善手をピンポイントで特定できていることを意味する。
    
    \item \textbf{クラスタ一致率(Cluster Agreement Rate)}: \\
    AllGrid MCTSが選択した候補手が、Clustered MCTSが選択した候補手と同じ「クラスタ」に属している割合。
    カーリングにおいては、数cm単位の厳密な座標の一致よりも、「ガードの裏に隠す」「ハウス内の特定のエリアに置く」といった戦術的な意図が合致しているかが重要である。したがって、完全一致率が低くても、クラスタ一致率が高ければ、戦略的に妥当な判断ができていると評価できる。
    
\end{itemize}

\subsection{実験条件}
評価実験には、実際の試合で頻出する様々な局面や戦略的判断が分かれる局面を想定したテストケースを用いた。
具体的には、以下の10種類のジャンルを用意し、各ジャンルにつき配置を微調整した10通りのバリエーションを作成した。合計100ケース(10ジャンル $\times$ 10バリエーション)の局面に対して実験を行った。

使用した局面パターンのジャンルは以下の通りである。
\begin{enumerate}
    \item \textbf{CenterGuard}: センターライン上にガードストーンが配置された状況。ハウス中心への直通コースを塞ぐことで、相手のドローショットを牽制したり、自チームが後に中心を攻めるための拠点とする防御的な局面である。
    \item \textbf{CornerGuards}:  ハウスの端(コーナー)を守る位置にガードストーンが配置された状況。ハウス中心を空けておくことで複数得点を狙う攻撃的な展開(ブランクエンド狙いなど)への布石となる局面である。
    \item \textbf{SingleDraw}: ハウス内にストーンが1つだけ存在するシンプルな配置。そのストーンに対するテイクアウトや、フリーズ、あるいは無視して別の場所に置くなど、基礎的な着手選択の精度が問われる局面である。
    \item \textbf{DoubleDraw}: ハウス内に2つのストーンが配置された状況。ダブルテイクアウト(2つ同時に弾き出す)を狙うか、片方だけを処理するかなど、複数の石に対する相互作用を考慮する必要がある局面である
    \item \textbf{HouseCorners}: ウス内の四隅にストーンが散らばっている状況。中心から離れた位置にあるストーンに対して、曲がり幅の大きいショットや正確なウェイト配分でアプローチできるかが試される局面である。
    \item \textbf{GuardAndDraw}: ハウス前のガードストーンとハウス内のドローストーンが組み合わさった配置。ガードを回避して中の石を狙うカムアラウンドショットや、ガードを利用したランバックショットなど、高度なライン取りが要求される局面である。
    \item \textbf{Random}: シート全域にランダムにストーンを配置した状況。定石や典型的なパターンには当てはまらない予期せぬ局面において、汎用的な対応能力や探索のロバスト性を検証するための局面である。
    \item \textbf{FreezeAttempt}: 相手のストーンに対してフリーズ(接触させて止める)ショットが有効となる配置。相手の石を壁として利用し、自らの石をテイクアウトから守るような、高精度な配置戦略が求められる局面である。
    \item \textbf{Corner}:  ハウスの左右の端付近で攻防が行われている配置。センターストーンとは異なり、石が外に滑り出やすい位置であるため、非常に繊細なウェイトコントロールとカール制御が重要となる局面である。
    \item \textbf{Crowded}: ハウス内外に多数のストーンが密集している混戦状態。一度のショットで複数の石が動く可能性が高く、物理挙動の連鎖(玉突き)を正確に予測する深い読みが必要となる局面である。
\end{enumerate}

\begin{figure}[H]
    \centering
    \begin{minipage}{0.19\textwidth}
        \centering
        \includegraphics[width=\linewidth]{figure/CenterGuard_v0.png}
        \caption{Center\-Guard}
    \end{minipage}
    \hfill
    \begin{minipage}{0.19\textwidth}
        \centering
        \includegraphics[width=\linewidth]{figure/CornerGuards_v0.png}
        \caption{Corner\-Guards}
    \end{minipage}
    \hfill
    \begin{minipage}{0.19\textwidth}
        \centering
        \includegraphics[width=\linewidth]{figure/SingleDraw_v0.png}
        \caption{Single\-Draw}
    \end{minipage}
    \hfill
    \begin{minipage}{0.19\textwidth}
        \centering
        \includegraphics[width=\linewidth]{figure/DoubleDraw_v0.png}
        \caption{Double\-Draw}
    \end{minipage}
    \hfill
    \begin{minipage}{0.19\textwidth}
        \centering
        \includegraphics[width=\linewidth]{figure/HouseCorners_v0.png}
        \caption{House\-Corners}
    \end{minipage}
\end{figure}

\begin{figure}[H]
    \centering
    \begin{minipage}[t]{0.19\textwidth}
        \centering
        \includegraphics[width=\linewidth]{figure/GuardAndDraw_v6.png}
        \caption{Guard\-And\-Draw}
    \end{minipage}
    \hfill
    \begin{minipage}[t]{0.19\textwidth}
        \centering
        \includegraphics[width=\linewidth]{figure/Random_v6.png}
        \caption{Random}
    \end{minipage}
    \hfill
    \begin{minipage}[t]{0.19\textwidth}
        \centering
        \includegraphics[width=\linewidth]{figure/FreezeAttempt_v6.png}
        \caption{Freeze\-Attempt}
    \end{minipage}
    \hfill
    \begin{minipage}[t]{0.19\textwidth}
        \centering
        \includegraphics[width=\linewidth]{figure/Corner_v0.png}
        \caption{Corner}
    \end{minipage}
    \hfill
    \begin{minipage}[t]{0.19\textwidth}
        \centering
        \includegraphics[width=\linewidth]{figure/Crowded_v0.png}
        \caption{Crowded}
    \end{minipage}
\end{figure}

\subsection{AllGrid v.s. Clusteredの対比実験}
各テストケースにおいて、AllGrid MCTSには十分な探索回数を与えて実行し、その結果を基準解とした。これに対し、Clustered MCTSは探索回数(イテレーション数)を段階的に変化させて実行し、少ない探索回数でどの程度基準解に近づけるか(一致率の推移)を検証した。具体的には、全探索に必要とされる探索回数の10\%から30\%程度の探索回数で実験を行った。
実験は全て同一の物理シミュレータ設定の下で行い、偶然性を排除するため各3回ずつ試行回数を確保した上で平均的な挙動を分析した。

\section{クラスタリングの実験結果}
図\ref{fig:overall_agreement}に、全100テストケースの平均クラスタ一致率の推移を示す。横軸は全100種テストケースのid、縦軸はAllGrid MCTS(基準解)とのクラスタ一致率を表している。
グラフが示す通り、同じテストジャンルであっても、テストケースごとには一致率の差分が生じている。

\begin{figure}[H]
    \centering
    \includegraphics[width=0.8\linewidth]{figure/all_100_cases_agreement_no_zoomin_repeat3.png}
    \caption{全100ケースにおける一致率の推移}
    \label{fig:overall_agreement}
\end{figure}

また、図\ref{fig:genre_agreement}に、10種類のジャンルごとのクラスタ一致率の推移を示す。
ジャンルによって一致率の上昇傾向に差異が見られる。「SingleDraw」のような比較的単純な局面では、少ない探索回数でも高い一致率に達する一方、「Crowded」の場合では、一致率の収束に多くの探索を要する傾向があることが読み取れる。

\begin{figure}[H]
    \centering
    \includegraphics[width=0.8\linewidth]{figure/genre_agreement_comparison_no_zoomin.png}
    \caption{ジャンル別のクラスタ一致率の推移}
    \label{fig:genre_agreement}
\end{figure}

一方で、図\ref{fig:genre_agreement}に、10種類のジャンルごとの完全一致率(Exact Agreement)の推移を示す。
これを見ると、探索回数を重ねても平均一致率は20\%程度に留まっており、AllGrid MCTSが導き出した最善手と完全に一致する確率が低いことが分かる。これは、クラスタリングによって候補手を代表点(medoid)に集約した際に、微細な座標に関する詳細な情報が失われてしまっているためであると考えられる。
すなわち、クラスタリングのみでは「大まかな方向性」は正しく捉えられていても、「真の最適解」をピンポイントで特定するには限界があることが示唆された。


\section{Zoom-In探索による解の精密化}
第2節の手法により、数千通りの候補手を数十個の「クラスタ代表手」に絞り込むことができた。しかし、これらの代表手はあくまでクラスタ内の中心点(Medoid)であり、真の最適解(最も得点期待値が高い微細な座標)とは限らない。代表手は「このあたりが良い」という大まかな方向性を示しているに過ぎない。

そこで本手法では、探索の後半フェーズとして「Zoom-In探索」を導入する。これは、MCTSによって最も有望である(訪問回数が最も多い)と判断されたクラスタに対して、その内部を詳細に再探索するプロセスである。

具体的な手順は以下の通りである。

\begin{enumerate}
    \item \textbf{有望クラスタの特定}: \\
    MCTSの探索終了後、ルートノード直下で最も訪問回数が多い子ノード(代表手)を特定する。このノードに対応するクラスタを「有望クラスタ」とする。

    \item \textbf{元の候補手の展開}: \\
    有望クラスタに所属していたメンバーである元の詳細なグリッド点をすべて復元する。例えば、あるクラスタが100個の類似した着手点から構成されていた場合、その100個すべてを再評価の対象とする。

    \item \textbf{集中探索(再シミュレーション)}: \\
    復元された詳細な候補手に対して、再度シミュレーション(または短いMCTS)を実行する。探索範囲が限定されているため、限られた計算リソース内でも各候補手に対して十分な数の試行を行うことができる。

    \item \textbf{最終的な着手の決定}: \\
    集中探索の結果、最もスコアが高かった候補手を、最終的な実行手として選択する。
\end{enumerate}

この「粗い探索(クラスタリング)」と「密な探索(Zoom-In)」の二段階方式により、広大な探索空間を効率的に絞り込みつつ、最終的なショットの精度を担保することが可能となる。

\section{Zoom-In探索の実験結果}
本節では、Zoom-In探索を適用した場合の一致率について評価する。Zoom-In探索は、クラスタリングによって選出された有望クラスタ内の候補手に対して集中的に再探索を行うことで、最終的な着手精度の向上を図る手法である。

図\ref{fig:zoomin_overall_agreement}に、Zoom-In探索適用後の全100テストケースにおける一致率を示す。クラスタリングのみの場合(図\ref{fig:overall_agreement})と比較すると、多くのテストケースにおいて一致率の向上が確認できる。これは、有望クラスタ内での集中探索により、より精密な候補手の選択が可能になったことを示している。

\begin{figure}[H]
    \centering
    \includegraphics[width=0.8\linewidth]{figure/all_100_cases_agreement_zoomin_repeat3.png}
    \caption{Zoom-In適用後の全100ケースにおける一致率}
    \label{fig:zoomin_overall_agreement}
\end{figure}

また、図\ref{fig:zoomin_genre_agreement}に、Zoom-In探索適用後のジャンル別一致率を示す。
クラスタリングのみの結果と比較すると、特に「Crowded」や「Random」といった複雑な局面において一致率の改善が顕著である。これは、複雑な局面ほど微細な座標の違いが結果に大きく影響するため、Zoom-In探索による精密化の効果が高いことを示唆している。

一方で、「SingleDraw」のような単純な局面では、Zoom-In探索の有無による差異は比較的小さい。これは、単純な局面ではクラスタリングの段階で既に適切な候補手が選出されており、追加の精密化の余地が限られているためと考えられる。

\begin{figure}[H]
    \centering
    \includegraphics[width=0.8\linewidth]{figure/genre_agreement_comparison_zoomin_repeat3.png}
    \caption{Zoom-In適用後のジャンル別一致率の比較}
    \label{fig:zoomin_genre_agreement}
\end{figure}

以上の結果から、提案手法である「クラスタリング+Zoom-In探索」の二段階方式は、限られた計算リソースの中で効率的に有効な候補手を探索するための有効なアプローチであることが確認された。


\chapter{まとめ}
研究のまとめ。なんやかんやなんやかんやなんやかんやなんやかんやなんやかんやなんやかんやなんやかんやなんやかんやなんやかんやなんやかんやなんやかんやなんやかんやなんやかんやなんやかんやなんやかんやなんやかんやなんやかんやなんやかんやなんやかんやなんやかんや

%=====================================================================================
% \chapter*{謝辞} %章を付けずにタイトル表示
% \addcontentsline{toc}{chapter}{謝辞} %章立てせずに目次に追加するおまじない
% 本論文を作成するにあたり、---- みなさまに感謝の意を表します.


%=====================================================================================

\addcontentsline{toc}{chapter}{参考文献} %章立てせずに目次に追加するおまじない
\renewcommand{\bibname}{参考文献} %これがないと,タイトルが「関連図書」になってしまう
\bibliographystyle{junsrt} %本文に\cite{}を入れることで,参考文献表示
\bibliography{refer} %bibtexファイルの読み込み


\end{document}

