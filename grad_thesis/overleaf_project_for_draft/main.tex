\documentclass[11pt,a4j]{jreport}

\usepackage{comment}
\usepackage{float}
\usepackage{color}
\usepackage{multicol}
\usepackage[dvipdfmx]{pict2e}
\usepackage{wrapfig}
\usepackage{graphicx}
\usepackage{bm}
\usepackage{url}
\usepackage{underscore}
\usepackage{colortbl}
\usepackage{tabularx}
\usepackage{fancyhdr}
\usepackage{ulem}
\usepackage{amsmath,amssymb,amsfonts}
\usepackage{algorithmic}
\usepackage{textcomp}
\usepackage{xcolor}
\usepackage[ipaex]{pxchfon}
\usepackage{algorithmic}
\usepackage{algorithm}
\usepackage{cite}

\usepackage[top=30truemm,bottom=30truemm,left=25truemm,right=25truemm]{geometry}
\setlength{\headheight}{15.5pt} % エラーで指定されている値よりも大きな値を設定する
\addtolength{\topmargin}{-3.5pt} % 必要に応じて topmargin も調整する


\begin{document}

\thispagestyle{empty}
\begin{center}

\vspace{20mm}
{\Large\noindent 2025年度 卒業論文}\\
\vspace{40mm}
{\huge\noindent\textbf{論文タイトル}}\\
\medskip
{\huge\noindent\textbf{論文タイトル(2行目)}}\\
\vspace{60mm}

%ここの空白文字は適当に
{\Large\noindent
2026年1月31日\\
\vspace{\baselineskip}
  所属 明治大学 \\
\vspace{\baselineskip}
指導教員 横山大作    \\
\vspace{\baselineskip}
 学籍番号 157R227127\\
\vspace{\baselineskip}
  名前 仲亜斗夢\\
}
\vspace{40mm}

\end{center}

\thispagestyle{empty}
\clearpage

%=====================================================================================
\renewcommand{\abstractname}{要旨}

\begin{abstract}
研究の要旨。なんやかんやなんやかんやなんやかんやなんやかんやなんやかんやなんやかんやなんやかんやなんやかんやなんやかんやなんやかんやなんやかんやなんやかんやなんやかんやなんやかんやなんやかんやなんやかんやなんやかんやなんやかんやなんやかんやなんやかんやなんなんや
\end{abstract}

%=====================================================================================

% 目次の表示
\tableofcontents

%=====================================================================================
\pagestyle{fancy}
\lhead{\rightmark}
\renewcommand{\chaptermark}[1]{\markboth{第\ \normalfont\thechapter\ 章~~#1}{}}
%=====================================================================================

\chapter{はじめに} %章


\section{研究背景} %1.1
\subsection{デジタルカーリングとは} %1.1.1


\section{研究目的}
近年、シミュレーション環境を用いた競技型AIの研究が盛んに行われており、特にデジタルカーリングのように行動空間が連続で構成されるゲームにおいては、高度な意思決定アルゴリズムが求められている。モンテカルロ木探索(Monte Carlo Tree Search, MCTS)は多くの分野で成功を収めているが、連続行動空間では探索の効率が著しく低下するという課題がある。特に大渡ら[1]の研究や加藤ら[2]の研究では、デジタルカーリングシステムにおけるゲーム木探索の有効性が示される一方で、候補手の絞り込みや、連続行動空間における最適な手の選択が依然として課題であることが明記されている。
一方、連続空間への対応としては深層強化学習も注目されているが、これらは学習コストが高く、設計も複雑であり、特定の環境に依存しやすいという欠点を持つ。従来のMCTSにおいても、グリッドによる固定的な離散化が試みられているが、局面ごとの柔軟性に欠け、探索精度を犠牲にしてしまう恐れがある。本研究はこれらの先行研究の課題を考慮したうえで、行動空間の離散化に加えてクラスタリングを導入し、局面に応じて動的に有効な候補手を抽出・選別することで、探索の深さと精度を両立する手法を提案する。これにより、限られた計算資源の中でも有効な手の選択が可能となり、デジタルカーリングAIの発展に寄与することを目的とする。

\section{本論文の構想}
近年、デジタルカーリングの分野ではMCTSを基盤としたAIプログラムが公式大会において高い競技成績を残しており、その有効性が実証されている。例えば、第10回UEC杯では大渡が開発したMCTSを用いた「歩」が優勝し[3]、さらに加藤らによって開発されたExpectimax-searchに基づく「じりつくん」シリーズは複数の大会で上位に入賞しており[4]、第11回大会では改良版である「Jiritsukun-Jr」が優勝を果たしている[5]。これらの成果は、MCTSがデジタルカーリングにおいて実践的かつ強力な手法であることを示している。
しかし、MCTSを連続行動空間であるデジタルカーリング環境[6]にそのまま適用するには課題が残されている。具体的には、無限に存在する候補手の中から探索対象を選定する必要があるため、計算資源が広く分散し、探索の効率が著しく低下してしまうという問題である。実際大渡らや加藤らの研究においても、候補手の絞り込みや行動の最適化が今後の課題として指摘されている。本研究では、連続行動空間を離散化し、さらにクラスタリングによって類似した行動を統合・代表化することで、効率よく有効な候補手を選定する手法を提案する。その有効性を検証するために、クラスタリングを導入したMCTS(以後、提案手法)と、クラスタリング無しの従来のMCTSとの比較実験を行う。評価は、①探索木の成長傾向、②候補手の選択分布、③対戦成績の3点から行う。
まず、各手法が生成する探索木を比較し、クラスタリングにより選ばれた代表候補手(集合K)のうち、実際に訪問回数の多かった手の割合を測定する。これにより、クラスタリングが有効な手を適切に抽出できたかを評価する。
次に、両手法を無制限時間下で対戦させ、その勝敗結果を性能指標とする。提案手法では、候補手集合Cの一部要素を代表手集合Kの要素に置き換えて、探索深さ3でプレイアウトを行い、選択された手の分布も記録する。
最後に、実際に両手法を対戦させ、提案手法が従来手法に比べて探索効率や精度の点で優れているかを評価し、その有効性を明らかにする。

\section{本論文の構成}


%これまでに調べた大事な技法は章単位で書いて良い
%ex. モンテカルロ, NN, 色彩について

\chapter{関連研究}
%調べたことは第1章よりもこっちの方が良い。
%研究目的に必要な情報以外はこっちに書くイメージ
[1]	大渡勝己, 田中哲朗:カーリングAIに対するモンテカルロ木探索の適用, ゲームプログラミングワークショップ2016論文集, pp. 180-187(2016).
[2]	加藤修, 飯塚博幸, 山本雅人:不確定性を含むデジタルカーリングにおけるゲーム木探索, 情報処理学会論文誌ジャーナル, Vol.57, No.11, pp.2354-2364(2016).
[4]	伊藤毅志, 森健太郎:デジタルカーリング大会報告2015年度, 情報処理学会研究報告, 2016-GI-36, No.2, pp. 1 - 5(2016).
[8]	佐藤佳州, 高橋大介:モンテカルロ木探索によるコンピュータ将棋, 情報処理学会論文誌, Vol.50, No.11, pp. 2740-2751(2019).

\section{デジタルカーリングに関する研究}
〇〇による先行研究\cite{uno}から××なことがわかる。
また、△△による先行研究\cite{wolf}から◇◇なこともわかる。

\section{MCTSに関する研究}

\chapter{提案手法}
\section{---}
\subsection{---}
\subsection{---}
\section{---}
\subsection{---}
\subsection{---}

\chapter{評価}
\section{評価方法}

\begin{figure}[tb]
\includegraphics[width=16cm]{figure/a.png}
\caption{AAA}
\label{fig:AAA}
\end{figure}

\begin{figure}[tb]
\includegraphics[width=16cm]{figure/b.jpg}
\caption{BBB}
\label{fig:BBB}
\end{figure}

\section{実験結果}

\section{考察}


\chapter{まとめ}
研究のまとめ。なんやかんやなんやかんやなんやかんやなんやかんやなんやかんやなんやかんやなんやかんやなんやかんやなんやかんやなんやかんやなんやかんやなんやかんやなんやかんやなんやかんやなんやかんやなんやかんやなんやかんやなんやかんやなんやかんやなんやかんや

%=====================================================================================
% \chapter*{謝辞} %章を付けずにタイトル表示
% \addcontentsline{toc}{chapter}{謝辞} %章立てせずに目次に追加するおまじない
% 本論文を作成するにあたり、---- みなさまに感謝の意を表します.


%=====================================================================================

\addcontentsline{toc}{chapter}{参考文献} %章立てせずに目次に追加するおまじない
\renewcommand{\bibname}{参考文献} %これがないと,タイトルが「関連図書」になってしまう
\bibliographystyle{junsrt} %本文に\cite{}を入れることで,参考文献表示
\bibliography{refer} %bibtexファイルの読み込み


\end{document}

